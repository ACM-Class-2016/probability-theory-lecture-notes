\input{def_Pascalprimer.tex}

\title{Some Thoughts}
\author{金之涵}
\date{}
\begin{document}
	\lstset{numbers=left,
		basicstyle=\scriptsize\courier,
		numberstyle=\tiny\courier\color{red!89!green!36!blue!36},
		language=C++,
		breaklines=true,
		keywordstyle=\color{blue!70},commentstyle=\color{red!50!green!50!blue!50},
		morekeywords={},
		stringstyle=\color{purple},
		frame=shadowbox,
		rulesepcolor=\color{red!20!green!20!blue!20}
	}
	\maketitle
	\newpage
	
	\begin{prob}
		For a sequence of random variables $X_1, X_2, \cdots$ with Markov property and a function $f$ on $X$. It is obvious that $f(X_1), f(X_2), \cdots$ are also random variables, however not necessarily holding Markov property.
		
		\sol
		
		In discrete cases, a function on $X$ is a process to merge some states just as what dynamic programming does, which requires the states with similar state transition rules.
		
		When $f$ is injective, Markov property must hold. Situation is not consistent on other cases. I only present a simple counterexample.
		
		Consider 2 random variables $X_1, X_2$ and the state universe is $X = \{1, 2, 3, 4\}$. The transition matrix $P$ and the function $f$ is following. Let $\mu = \{p_1, p_2, p_3, p_4\}$ be the distribution of $X_1$, then the distribution $\lambda$ of $X_2$ is $\{ 0, 0, p_1 + p_3, p_2 + p_4 \}$. Acted on $f$, $f(\lambda) = \{0, p_1 + p_3, p_2 + p_4\}$, which cannot be expressed as $f(\mu)P^\prime = \{p_1 + p_2, p_3, p_4\}P^\prime$. So $f(X_1), f(X_2)$ do not hold Markov property any more.
		
		\begin{equation*}
			P = {
				\left[ \begin{array}{cccc}
				0 & 0 & 1 & 0\\
				0 & 0 & 0 & 1\\
				0 & 0 & 1 & 0\\
				0 & 0 & 0 & 1
				\end{array} 
				\right ]}, \quad
			f(x)=\left\{
				\begin{aligned}
				1 &  & x = 1, 2 \\
				2 &  & x = 3 \\
				3 &  & x = 4
				\end{aligned}
				\right.
		\end{equation*}
		
		
	\end{prob}

\end{document}