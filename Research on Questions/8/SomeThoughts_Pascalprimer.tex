\documentclass[12pt, a4paper]{article}
\usepackage{ctex}

\usepackage[margin=1in]{geometry}
\usepackage{
  color,
  clrscode,
  amssymb,
  ntheorem,
  amsmath,
  listings,
  fontspec,
  xcolor,
  supertabular,
  multirow,
  mathtools,
}
\definecolor{bgGray}{RGB}{36, 36, 36}
\usepackage[
  colorlinks,
  linkcolor=bgGray,
  anchorcolor=blue,
  citecolor=green
]{hyperref}
\newfontfamily\courier{Courier}

\theoremstyle{margin}
\theorembodyfont{\normalfont}
\newtheorem{thm}{Theorem}
\newtheorem{cor}[thm]{Corollary}
\newtheorem{pos}[thm]{Proposition}
\newtheorem*{lemma}[thm]{Lemma}
\newtheorem{defi}[thm]{Definition}
\newtheorem{std}[thm]{Standard}
\newtheorem{imp}[thm]{Implimentation}
\newtheorem{alg}[thm]{Algorithm}
\newtheorem{exa}[thm]{Example}
\newtheorem{prob}[thm]{Problem}
\DeclareMathOperator{\sft}{E}
\DeclareMathOperator{\idt}{I}
\DeclareMathOperator{\spn}{span}
\DeclareMathOperator*{\agm}{arg\,min}
%\newcommand{\x}{\mbf{x}}
\newcommand{\pr}{\prime}
\newcommand{\tr}{^\intercal}
\newcommand{\st}{\text{s.t.}}
\newcommand{\hp}{^\prime}
\newcommand{\ms}{\mathscr}
\newcommand{\mn}{\mathnormal}
\newcommand{\tbf}{\textbf}
\newcommand{\mbf}{\mathbf}
\newcommand{\fl}{\mathnormal{fl}}
\newcommand{\f}{\mathnormal{f}}
\newcommand{\g}{\mathnormal{g}}
\newcommand{\R}{\mathbf{R}}
\newcommand{\Q}{\mathbf{Q}}
\newcommand{\JD}{\textbf{D}}
\newcommand{\rd}{\mathrm{d}}
\newcommand{\str}{^*}
\newcommand{\vep}{\varepsilon}
\newcommand{\lhs}{\text{L.H.S}}
\newcommand{\rhs}{\text{R.H.S}}
\newcommand{\con}{\text{Const}}
\newcommand{\oneton}{1,\,2,\,\dots,\,n}
\newcommand{\aoneton}{a_1a_2\dots a_n}
\newcommand{\xoneton}{x_1,\,x_2,\,\dots,\,x_n}
\newcommand\thmref[1]{Theorem~\ref{#1}}
\newcommand\lemmaref[1]{Lemma~\ref{#1}}
\newcommand\defref[1]{Definition~\ref{#1}}
\newcommand\posref[1]{Proposition~\ref{#1}}
\newcommand\secref[1]{Section~\ref{#1}}
\newcommand\equref[1]{(\ref{#1})}
\newcommand\figref[1]{Figure \ref{#1}}
\newcommand\corref[1]{Corollary~\ref{#1}}
\newcommand\exaref[1]{Example~\ref{#1}}
\newcommand\algref[1]{Algorithm~\ref{#1}}
\newcommand{\remark}{\paragraph{Remark}}
\newcommand{\example}{\paragraph{Example}}
\newcommand{\proof}{\paragraph{Proof}}
\newcommand{\sol}{\paragraph{Solution}}
\newcommand{\home}{\paragraph{Homework}}
\newcommand{\problemtitle}[1]{\gdef\@problemtitle{#1}}% Store problem title
\newcommand{\probleminput}[1]{\gdef\@probleminput{#1}}% Store problem input
\renewcommand\refname{Reference}
\newcommand{\problemquestion}[1]{\gdef\@problemquestion{#1}}% Store problem question
\usepackage{tabularx,lipsum,environ,amsmath,amssymb}

\makeatletter
\NewEnviron{problem}{
	\problemtitle{}\probleminput{}\problemquestion{}% Default input is empty
	\BODY% Parse input
	\par\addvspace{.5\baselineskip}
	\noindent
	\begin{tabularx}{\textwidth}{@{\hspace{\parindent}} l X c}
		\multicolumn{2}{@{\hspace{\parindent}}l}{\@problemtitle} \\% Title
		\textbf{Input:} & \@probleminput \\% Input
		\textbf{Question:} & \@problemquestion% Question
	\end{tabularx}
	\par\addvspace{.5\baselineskip}
}

\title{Some Thoughts}
\author{金之涵}
\date{}
\begin{document}
	\lstset{numbers=left,
		basicstyle=\scriptsize\courier,
		numberstyle=\tiny\courier\color{red!89!green!36!blue!36},
		language=C++,
		breaklines=true,
		keywordstyle=\color{blue!70},commentstyle=\color{red!50!green!50!blue!50},
		morekeywords={},
		stringstyle=\color{purple},
		frame=shadowbox,
		rulesepcolor=\color{red!20!green!20!blue!20}
	}
	\maketitle
	\newpage
	
	\begin{prob}
		For a sequence of random variables $X_1, X_2, \cdots$ with Markov property and a function $f$ on $X$. It is obvious that $f(X_1), f(X_2), \cdots$ are also random variables, however not necessarily holding Markov property.
		
		\sol
		
		In discrete cases, a function on $X$ is a process to merge some states just as what dynamic programming does, which requires the states with similar state transition rules.
		
		When $f$ is injective, Markov property must hold. Situation is not consistent on other cases. I only present a simple counterexample.
		
		Consider 2 random variables $X_1, X_2$ and the state universe is $X = \{1, 2, 3, 4\}$. The transition matrix $P$ and the function $f$ is following. Let $\mu = \{p_1, p_2, p_3, p_4\}$ be the distribution of $X_1$, then the distribution $\lambda$ of $X_2$ is $\{ 0, 0, p_1 + p_3, p_2 + p_4 \}$. Acted on $f$, $f(\lambda) = \{0, p_1 + p_3, p_2 + p_4\}$, which cannot be expressed as $f(\mu)P^\prime = \{p_1 + p_2, p_3, p_4\}P^\prime$. So $f(X_1), f(X_2)$ do not hold Markov property any more.
		
		\begin{equation*}
			P = {
				\left[ \begin{array}{cccc}
				0 & 0 & 1 & 0\\
				0 & 0 & 0 & 1\\
				0 & 0 & 1 & 0\\
				0 & 0 & 0 & 1
				\end{array} 
				\right ]}, \quad
			f(x)=\left\{
				\begin{aligned}
				1 &  & x = 1, 2 \\
				2 &  & x = 3 \\
				3 &  & x = 4
				\end{aligned}
				\right.
		\end{equation*}
		
		
	\end{prob}

\end{document}