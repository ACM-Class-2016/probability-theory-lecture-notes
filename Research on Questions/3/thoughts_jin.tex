\documentclass[12pt, a4paper]{article}
\usepackage{ctex}

\usepackage[margin=1in]{geometry}
\usepackage{
  color,
  clrscode,
  amssymb,
  ntheorem,
  amsmath,
  listings,
  fontspec,
  xcolor,
  supertabular,
  multirow,
  mathtools,
}
\definecolor{bgGray}{RGB}{36, 36, 36}
\usepackage[
  colorlinks,
  linkcolor=bgGray,
  anchorcolor=blue,
  citecolor=green
]{hyperref}
\newfontfamily\courier{Courier}

\theoremstyle{margin}
\theorembodyfont{\normalfont}
\newtheorem{thm}{Theorem}
\newtheorem{cor}[thm]{Corollary}
\newtheorem{pos}[thm]{Proposition}
\newtheorem*{lemma}[thm]{Lemma}
\newtheorem{defi}[thm]{Definition}
\newtheorem{std}[thm]{Standard}
\newtheorem{imp}[thm]{Implimentation}
\newtheorem{alg}[thm]{Algorithm}
\newtheorem{exa}[thm]{Example}
\newtheorem{prob}[thm]{Problem}
\DeclareMathOperator{\sft}{E}
\DeclareMathOperator{\idt}{I}
\DeclareMathOperator{\spn}{span}
\DeclareMathOperator*{\agm}{arg\,min}
%\newcommand{\x}{\mbf{x}}
\newcommand{\pr}{\prime}
\newcommand{\tr}{^\intercal}
\newcommand{\st}{\text{s.t.}}
\newcommand{\hp}{^\prime}
\newcommand{\ms}{\mathscr}
\newcommand{\mn}{\mathnormal}
\newcommand{\tbf}{\textbf}
\newcommand{\mbf}{\mathbf}
\newcommand{\fl}{\mathnormal{fl}}
\newcommand{\f}{\mathnormal{f}}
\newcommand{\g}{\mathnormal{g}}
\newcommand{\R}{\mathbf{R}}
\newcommand{\Q}{\mathbf{Q}}
\newcommand{\JD}{\textbf{D}}
\newcommand{\rd}{\mathrm{d}}
\newcommand{\str}{^*}
\newcommand{\vep}{\varepsilon}
\newcommand{\lhs}{\text{L.H.S}}
\newcommand{\rhs}{\text{R.H.S}}
\newcommand{\con}{\text{Const}}
\newcommand{\oneton}{1,\,2,\,\dots,\,n}
\newcommand{\aoneton}{a_1a_2\dots a_n}
\newcommand{\xoneton}{x_1,\,x_2,\,\dots,\,x_n}
\newcommand\thmref[1]{Theorem~\ref{#1}}
\newcommand\lemmaref[1]{Lemma~\ref{#1}}
\newcommand\defref[1]{Definition~\ref{#1}}
\newcommand\posref[1]{Proposition~\ref{#1}}
\newcommand\secref[1]{Section~\ref{#1}}
\newcommand\equref[1]{(\ref{#1})}
\newcommand\figref[1]{Figure \ref{#1}}
\newcommand\corref[1]{Corollary~\ref{#1}}
\newcommand\exaref[1]{Example~\ref{#1}}
\newcommand\algref[1]{Algorithm~\ref{#1}}
\newcommand{\remark}{\paragraph{Remark}}
\newcommand{\example}{\paragraph{Example}}
\newcommand{\proof}{\paragraph{Proof}}
\newcommand{\sol}{\paragraph{Solution}}
\newcommand{\home}{\paragraph{Homework}}
\newcommand{\problemtitle}[1]{\gdef\@problemtitle{#1}}% Store problem title
\newcommand{\probleminput}[1]{\gdef\@probleminput{#1}}% Store problem input
\renewcommand\refname{Reference}
\newcommand{\problemquestion}[1]{\gdef\@problemquestion{#1}}% Store problem question
\usepackage{tabularx,lipsum,environ,amsmath,amssymb}

\makeatletter
\NewEnviron{problem}{
	\problemtitle{}\probleminput{}\problemquestion{}% Default input is empty
	\BODY% Parse input
	\par\addvspace{.5\baselineskip}
	\noindent
	\begin{tabularx}{\textwidth}{@{\hspace{\parindent}} l X c}
		\multicolumn{2}{@{\hspace{\parindent}}l}{\@problemtitle} \\% Title
		\textbf{Input:} & \@probleminput \\% Input
		\textbf{Question:} & \@problemquestion% Question
	\end{tabularx}
	\par\addvspace{.5\baselineskip}
}

\title{Some Thoughts and Solutions}
\author{金之涵}
\date{}

\begin{document}
	\lstset{numbers=left,
		basicstyle=\scriptsize\courier,
		numberstyle=\tiny\courier\color{red!89!green!36!blue!36},
		language=C++,
		breaklines=true,
		keywordstyle=\color{blue!70},commentstyle=\color{red!50!green!50!blue!50},
		morekeywords={},
		stringstyle=\color{purple},
		frame=shadowbox,
		rulesepcolor=\color{red!20!green!20!blue!20}
	}
	\maketitle
	\newpage
	
	\begin{prob} [Sunrise Problem]
		Consider we know nothing about sunrise but the fact that the sun has risen once a day for $N$ days, what is the probability of the sun also rising tomorrow? Because we have no idea the probability $p$ of the sun rising on any given day, we only the situation with $p$ uniformly distributed in $[0, 1]$.
	
		\sol
		Let A be the event that the sun rises tomorrow and B be the event that the sun has risen once a day during the past $N$ days. Similar to Bayes' law in discrete form, we have the following equation, where $dp$ is the distribution of $p$.
		
		\begin{equation*}
			\begin{array}{cc}
				P(A|B)  &= \dfrac{P(A \cap B)}{P(B)} \\
						&= \dfrac{\int\limits_{0}^{1}p^{N + 1} dp}{\int\limits_{0}^{1}p^{N} dp} \\
						&= \dfrac{N + 1}{N + 2}
			\end{array}
		\end{equation*}
		
	\end{prob}

	\newpage
	
	\begin{prob} [Matrix Test]
	Assume matrix $A$, $B$, $C$ are uniformly distributed in $\mathbb{F}_2^{n \times n}$ independently, which means each element in $A$, $B$, $C$ is uniformly distributed in $\mathbb{F}_2$ independently. Consider a method to test if $AB = C$ that we generate a random vector r $\in \mathbb{F}_2^{n \times 1}$ and determine the result by calculating $(AB - C)r$. Although this new method is more efficient, we want to know its precision.
	
	\sol
	Let $P$ be the event that $AB = C$, $Q$ be the event that $ABr = Cr$. Due to the fact that $ABr = Cr$ always holds when $AB = C$, we only care about the situation when $AB \neq C$, which is $P(Q|P^c)$.
	
	\begin{equation*}
		P(Q|P^c) = \dfrac{P(Q \cap P^c)}{P(P^c)} \
				= \dfrac{\sum\limits_{i = 1}^{n}P(r(AB - C) = i) \cdot P(Q | r(AB - C) = i)}{P(P^c)}
	\end{equation*}
	
	Randomly generating $A$, $B$ and $C$ in order, we can find that the distribution of $AB - C$ is the same as that of $C$. So we can replace $AB - C$ with $C$.
	
%	\begin{align*}
		\begin{equation*}
		\begin{split}
			P(P^c) = P(C \neq 0) &= 1 - 2^{-n^2} \\
			P(r(AB - C) = i) &= P(r(C) = i)
		\end{split}
		\end{equation*}
%	\end{align*}
	
	We then consider the kernel of $C$.
	
	\begin{equation*}
		Cr = 0 \iff r \in Ker(C)
	\end{equation*}

	So $P(Q | r(AB - C) = i) = 2 ^ {-i}$. We now have the following equation.
	
	\begin{equation*}
		P(Q|P^c) = \dfrac{1}{1 - 2^{-n^2}} \cdot \sum\limits_{i = 1}^{n}P(r(C) = i) \cdot 2^{-i}
	\end{equation*}
	
	A rough estimate can be obtained as following using $2^{-i} \le 2^{-1}$. The test can be considered reliable after testing several times with different $r$.
		
	\begin{equation}
		P(Q|P^c) \le \frac{1}{2}
	\end{equation}
	
	We will next use a lemma to acquire a estimate close to the actual situation. 
		
	\begin{lemma}
		Let $B_{n, k}$ be the number of ordered $k$-basis of a subspace of $\mathbb{F}_2^{n}$.\cite{7762210}
		\begin{equation*}
			B_{n, k} = \prod\limits_{i = 0}^{k - 1} (2 ^ n - 2 ^ i)
		\end{equation*}
	\end{lemma}

	\proof
	
	Every linear space has its basis. There are $(2 ^ n - 2 ^ 0)$ vectors to choose from for the first element, $(2 ^ n - 2 ^ 1)$ to choose from for the second element, $\cdots$ $(2 ^ n - 2 ^ {k - 1})$ vectors to choose from for the $k$th element.
	
	$\hfill\blacksquare$ 
	
	Let $f_{n, k}$ be the number of matrices($\in \mathbb{F}_2^{n \times n}$) whose rank is $k$. We will next count $f_{n, k}$ by two steps.
	
	First, count the number of linear spaces of matrix $M \in \mathbb{F}_2^{n \times n}$ whose rank is $k$. A linear space is determined by a ordered basis $v_1, v_2, \cdots, v_k$, which has $B_{n, k}$ cases. However each space is counted $B_{k, k}$ times. So the number of linear spaces of $M$ is $\dfrac{B_{n, k}}{B_{k, k}}$.
	
	Second, count the number of matrices that forms a identical subspace of $\mathbb{F}_2^n$. Let $U$ be a fixed subspace and $R$ be a fixed $k \times n$ matrix whose row vectors form a basis of $U$. Let $A$ be any matrix that forms $U$. Since each row vector of $A$ can be expressed uniquely as a linear combination of rows of $R$, there exists a unique $n \times k$ matrix $M$ such that $A = MR$. Obviously rank($A$) is $k$. On the other hand, for any $A_{n \times n}$ with rank $k$ forming $U$, $A$ can be factorized as $A_{n \times n} = M_{n \times k}R_{k \times n}$, where rank($M$) is $k$. So $A$ is only determined by $R$, the number of valid $A$s is $B_{n, k}$.
	
	\begin{equation*}
		f_{n, k} = \dfrac{B_{n, k} ^ 2}{B_{k, k}}
	\end{equation*}
	
	It is easy to see 
	
	\begin{align*}\label{inequ}
		\dfrac{\dfrac{f_{n + 1, k + 1} \cdot 2 ^ {-(k + 1)}}{2^{(n + 1) ^ 2 - 1}}}
				{\dfrac{f_{n, k} \cdot 2 ^ {-k}}{2 ^ {n ^ 2 - 1}}} &= 
			\dfrac{1}{2} \cdot \dfrac{{2 ^ {n ^ 2}} - 1}{2 ^ {(n + 1) ^ 2} - 1} \cdot
				\dfrac{{(2 ^ {n + 1} - 1)} ^ 2 \cdot 2 ^ k}{2 ^ {k + 1} - 1} \\
			&< \dfrac{1}{2} \cdot \dfrac{1}{2 ^ {(n + 1) ^ 2 - n ^ 2}} \cdot 
				\dfrac{1}{2} \cdot 2 ^ {2n + 2} \\
			&< \dfrac{1}{2}
	\end{align*}
	
	The probability of dimention $n$ can be expressed as
	
	\begin{equation*}
		P_n(Q|P^c) = \dfrac{1}{2^{n^2} - 1} \cdot \sum\limits_{i = 1}^{n}f_{n, i} \cdot 2^{-i}
	\end{equation*}
	
	\begin{align*}
		\dfrac{P_{n + 1}(Q|P^c)}{P_{n}(Q|P^c)}
		 &= \dfrac{\dfrac{1}{2^{(n + 1)^2} - 1} \cdot \sum\limits_{i = 1}^{n + 1}f_{n + 1, i} \cdot 2^{-i}}
		 			{\dfrac{1}{2^{n^2} - 1} \cdot \sum\limits_{i = 1}^{n}f_{n, i} \cdot 2^{-i}} \\ 
		 &<= \dfrac{2^{n^2} - 1}{2^{(n + 1) ^ 2} - 1} \cdot \dfrac{f_{n + 1, 1} \cdot
		 	 \dfrac{1}{2}}{\sum\limits_{i = 1}^{n}f_{n, i} \cdot 2^{-i}} + 
	 	 	 \max_{k = 1}^{n} \dfrac{\dfrac{f_{n + 1, k + 1} \cdot 2 ^ {-(k + 1)}}
	 	 	 	{2^{(n + 1) ^ 2 - 1}}}{\dfrac{f_{n, k} \cdot 2 ^ {-k}}{2 ^ {n ^ 2 - 1}}} \ref{inequ} \\
 	 	 &< \dfrac{1}{2 ^ {2n + 2}} \cdot \dfrac{(2 ^ {n + 1} - 1) ^ 2}
 	 	 	{2^{-n} \cdot \sum\limits_{i = 1}^{n}f_{n, i}} + \dfrac{1}{2} \\
 	 	 &< \dfrac{1}{2^{2n + 2}} \cdot \dfrac{2^{n + 1} - 1}{2 ^ {-n}} \cdot 
 	 	 	\dfrac{2 ^ {n + 1} - 1}{2^{n^2} - 1} + \dfrac{1}{2}\\
 	 	 &< \dfrac{1}{2^{2n + 2}} \cdot \dfrac{2^{n + 1}}{2 ^ {-n}} \cdot 
 	 	 \dfrac{2 ^ {n + 1}}{2^{n^2}} + \dfrac{1}{2}\\
 	 	 &< \dfrac{1}{2} + \epsilon
	\end{align*}
	
	where $\lim\limits_{n \to \infty} \epsilon = 0$. So $P_n(Q|P^c)$ can be expressed as a geometric sequence form, which decreases fast as the increasing of $n$.
	
	\begin{equation}
		P_n(Q|P^c) < C \cdot c^n, 0 < c < 1
	\end{equation}
	
	After running a code, we have the following data.
	
	\begin{align*}
		\begin{tabular}{|c|c|c|c|c|c|c|c|c|}
		\hline
		n & 1 & 2& 3 & 4 & 5 & 6 & 7 & 8\\
		\hline
		$P_n(Q|P^c)$  & 0.5 & 0.4 & 0.23288 & 0.12108 & 0.06152 & 0.03101 & 0.01556 & 0.00780\\
		\hline
		$\frac{P_{n + 1}(Q|P^c)}{P_n(Q|P^c)}$  & 0.8 & 0.5822 & 0.51993 & 0.50812 & 0.50397 & 0.50197 & 0.50098 & 0.50049\\
		\hline
		\end{tabular}
	\end{align*}
	
	The actual trend of the $P_n$ is close to a geometry sequence with common ratio 0.5, when $n > 2$. So the test is reliable when $n$ is large, even if testing only one time.
	
	\end{prob}
	
	\newpage
	
	\bibliographystyle{unsrt}
	\bibliography{ref}
\end{document}