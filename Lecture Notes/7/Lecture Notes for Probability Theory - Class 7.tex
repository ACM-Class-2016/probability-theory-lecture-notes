\documentclass[12pt, a4paper]{article}
\usepackage{ctex}

\usepackage[margin=1in]{geometry}
\usepackage{
  color,
  clrscode,
  amssymb,
  ntheorem,
  amsmath,
  listings,
  fontspec,
  xcolor,
  supertabular,
  multirow,
  mathtools,
}
\definecolor{bgGray}{RGB}{36, 36, 36}
\usepackage[
  colorlinks,
  linkcolor=bgGray,
  anchorcolor=blue,
  citecolor=green
]{hyperref}
\newfontfamily\courier{Courier}

\theoremstyle{margin}
\theorembodyfont{\normalfont}
\newtheorem{thm}{Theorem}
\newtheorem{cor}[thm]{Corollary}
\newtheorem{pos}[thm]{Proposition}
\newtheorem*{lemma}[thm]{Lemma}
\newtheorem{defi}[thm]{Definition}
\newtheorem{std}[thm]{Standard}
\newtheorem{imp}[thm]{Implimentation}
\newtheorem{alg}[thm]{Algorithm}
\newtheorem{exa}[thm]{Example}
\newtheorem{prob}[thm]{Problem}
\DeclareMathOperator{\sft}{E}
\DeclareMathOperator{\idt}{I}
\DeclareMathOperator{\spn}{span}
\DeclareMathOperator*{\agm}{arg\,min}
%\newcommand{\x}{\mbf{x}}
\newcommand{\pr}{\prime}
\newcommand{\tr}{^\intercal}
\newcommand{\st}{\text{s.t.}}
\newcommand{\hp}{^\prime}
\newcommand{\ms}{\mathscr}
\newcommand{\mn}{\mathnormal}
\newcommand{\tbf}{\textbf}
\newcommand{\mbf}{\mathbf}
\newcommand{\fl}{\mathnormal{fl}}
\newcommand{\f}{\mathnormal{f}}
\newcommand{\g}{\mathnormal{g}}
\newcommand{\R}{\mathbf{R}}
\newcommand{\Q}{\mathbf{Q}}
\newcommand{\JD}{\textbf{D}}
\newcommand{\rd}{\mathrm{d}}
\newcommand{\str}{^*}
\newcommand{\vep}{\varepsilon}
\newcommand{\lhs}{\text{L.H.S}}
\newcommand{\rhs}{\text{R.H.S}}
\newcommand{\con}{\text{Const}}
\newcommand{\oneton}{1,\,2,\,\dots,\,n}
\newcommand{\aoneton}{a_1a_2\dots a_n}
\newcommand{\xoneton}{x_1,\,x_2,\,\dots,\,x_n}
\newcommand\thmref[1]{Theorem~\ref{#1}}
\newcommand\lemmaref[1]{Lemma~\ref{#1}}
\newcommand\defref[1]{Definition~\ref{#1}}
\newcommand\posref[1]{Proposition~\ref{#1}}
\newcommand\secref[1]{Section~\ref{#1}}
\newcommand\equref[1]{(\ref{#1})}
\newcommand\figref[1]{Figure \ref{#1}}
\newcommand\corref[1]{Corollary~\ref{#1}}
\newcommand\exaref[1]{Example~\ref{#1}}
\newcommand\algref[1]{Algorithm~\ref{#1}}
\newcommand{\remark}{\paragraph{Remark}}
\newcommand{\example}{\paragraph{Example}}
\newcommand{\proof}{\paragraph{Proof}}
\newcommand{\sol}{\paragraph{Solution}}
\newcommand{\home}{\paragraph{Homework}}
\newcommand{\problemtitle}[1]{\gdef\@problemtitle{#1}}% Store problem title
\newcommand{\probleminput}[1]{\gdef\@probleminput{#1}}% Store problem input
\renewcommand\refname{Reference}
\newcommand{\problemquestion}[1]{\gdef\@problemquestion{#1}}% Store problem question
\usepackage{tabularx,lipsum,environ,amsmath,amssymb}

\makeatletter
\NewEnviron{problem}{
	\problemtitle{}\probleminput{}\problemquestion{}% Default input is empty
	\BODY% Parse input
	\par\addvspace{.5\baselineskip}
	\noindent
	\begin{tabularx}{\textwidth}{@{\hspace{\parindent}} l X c}
		\multicolumn{2}{@{\hspace{\parindent}}l}{\@problemtitle} \\% Title
		\textbf{Input:} & \@probleminput \\% Input
		\textbf{Question:} & \@problemquestion% Question
	\end{tabularx}
	\par\addvspace{.5\baselineskip}
}

\title{Lecture Notes for Probability Theory - Class 7}
\author{Yuwei Wu}
\date{}
\begin{document}
	\lstset{numbers=left,
		basicstyle=\scriptsize\courier,
		numberstyle=\tiny\courier\color{red!89!green!36!blue!36},
		language=C++,
		breaklines=true,
		keywordstyle=\color{blue!70},commentstyle=\color{red!50!green!50!blue!50},
		morekeywords={},
		stringstyle=\color{purple},
		frame=shadowbox,
		rulesepcolor=\color{red!20!green!20!blue!20}
	}
	\maketitle
	\newpage
	\begin{thm} [MacMillan Theorem]
		\mbox{}\par
		\noindent 设集合 $S:=\{x_1,x_2,...,x_r\}$,在$S^n$上的有限样本空间$\Omega_n  = \{ \omega  = ({\omega _1},{\omega _2},...,\omega_n)\}$,其中$\omega_k$都是互相独立且服从相同分布$P_S$从$S$中取值的。令$P_i=P_S(x_i)$,记关于分布$P_S$的熵${H} :=  - \sum\limits_{i = 1}^r {{P_i}\log ({P_i})}$。\\
		在概率空间$(\Omega_n,P)$中,对于任意$\varepsilon > 0$,在$n$足够大时,总能找到$\Omega_n^`\subseteq \Omega_n$,使得:\\
		1. $exp(n(H-\varepsilon)) \le |\Omega_n^`| \le exp(n(H+\varepsilon))$\\
		2. $\mathop {\lim }\limits_{n \to \infty } P(\Omega _n^`) = 1$\\
		3. 对于每一个$\omega \subseteq \Omega_n^`$,有$exp(-n(H+\varepsilon)) \le P(\omega)\le exp(-n(H-\varepsilon))$\\
		\proof
		\mbox{}\par
		\noindent 首先,由结论(2)(3)可推得(1):\\
		由于$(2)\mathop {\lim }\limits_{n \to \infty } P(\Omega _n^`) =1$,故对任意$\varepsilon_0>0$总存在$N_0(\varepsilon_0))$使得$N>N_0$时,$|P(\Omega_N^`)-1|<\varepsilon_0$,即$$1-\varepsilon_0<P(\Omega_N^`)<1+\varepsilon_0$$\\
		又由于$P({\Omega _N}^`) = \sum\limits_{\omega  \in {\Omega _N}^`} {P(\omega )} $,故$|{\Omega _N}^`| \cdot P{(\omega )_{\min }} \le P({\Omega _N}^`) \le |{\Omega _N}^`| \cdot P{(\omega )_{\max }}$,即$$\frac{{P({\Omega _N}^`)}}{{P{{(\omega )}_{\max }}}} \le |{\Omega _N}^`| \le \frac{{P({\Omega _N}^`)}}{{P{{(\omega )}_{\min }}}}$$\\
		再由(3)对每一个$\omega \subseteq \Omega_N^`$,有$exp(-n(H+\varepsilon)) \le P(\omega)\le exp(-n(H-\varepsilon))$,得\\
        $$P{(\omega )_{\min }}=exp(-n(H+\varepsilon)),P{(\omega )_{\max }}=exp(-n(H-\varepsilon))$$\\
        故对任意$\varepsilon_0>0$,总存在$N_0(\varepsilon_0)$使得$N>N_0(\varepsilon_0)$时,有
        $P({\Omega _N}^`) \cdot \exp (N(H - \varepsilon )) \le |{\Omega _N}^`| \le P({\Omega _N}^`) \cdot \exp (N(H + \varepsilon ))$,即对任意$\varepsilon_0>0$,总存在$N_0(\varepsilon_0)$使得$N>N_0$时,成立\\
        $$(1 - \varepsilon_0 ) \cdot \exp (N(H - \varepsilon )) \le |{\Omega _N}^`| \le (1 + \varepsilon_0 ) \cdot \exp (N(H + \varepsilon ))$$\\
        由于$\varepsilon_0$的任意性,易得$n$充分大时,成立(1)
        $$exp(n(H-\varepsilon)) \le |\Omega_n^`| \le exp(n(H+\varepsilon))$$
        于是只须证明(2)(3)。对于(2):\\
        对于任意$\varepsilon>0$,有$\delta=\frac{\varepsilon}{\sum\limits_{j = 1}^r logP_j}$,构造$\Omega_n^`=\{\omega \in \Omega_n:|\frac{|V_j(\omega)|}{n}-P_j| \le \delta,1\le j \le r\}$,其中$V_j(\omega)=\{i \in [n]:\omega_i=x_j\}$,\\
        由弱大数定律得:\[\forall 1 \leqslant j \leqslant r,\mathop {lim}\limits_{n \to \infty } P(|\frac{{{V_j}(\omega )}}{n} - {P_j}| > \delta ) = 0,\]
        于是$\mathop {\lim }\limits_{n \to \infty } P({\complement _{{\Omega _n}}}\Omega _n^`) = 0$,即$\mathop {\lim }\limits_{n \to \infty } P(\Omega _n^`) = 1$,(2)得证。\\
        下证上述构造的对于$\Omega _n^`$对(3)也成立:\\
        对于每一个$\omega = (\omega_1,\omega_2,...,\omega_n) \subseteq \Omega_n^`$,有
        \begin{align*}  
        P(\omega ) &= P({\omega _1}) \cdot P({\omega _2}) \ldots P({\omega _n}) \\  
        &= P_1^{|V_1(\omega)|}\cdot P_2^{|V_2(\omega)|} \ldots P_r^{|V_r(\omega)|}  \\  
        &= exp(\sum\limits_{j = 1}^r {V_j(\omega)\cdot logP_j}) \\
        &= exp(n\cdot \sum\limits_{j = 1}^r {\frac{V_j(\omega)}{n}\cdot logP_j}) 
        \end{align*}  
        由于$\omega \in \Omega_n^`$,故有${P_j} - \delta  \leqslant \frac{{{V_j}(\omega )}}{n} \leqslant {P_j} + \delta $,代入上式得到:
        \[\exp (n \cdot \sum\limits_{j = 1}^r {({P_j} - \delta } ) \cdot log{P_j}) \leqslant P(\omega ) \leqslant \exp (n \cdot \sum\limits_{j = 1}^r {({P_j} + \delta } ) \cdot log{P_j}),\]
        即\[\exp (n \cdot \sum\limits_{j = 1}^r {{P_j}}  \cdot log{P_j}) \cdot exp( - n\sum\limits_{j = 1}^r \delta   \cdot log{P_j}) \leqslant P(\omega ) \leqslant \exp (n \cdot \sum\limits_{j = 1}^r {{P_j}}  \cdot log{P_j}) \cdot exp(n\sum\limits_{j = 1}^r \delta   \cdot log{P_j}),\]
        故\[\exp ( - n(H + \varepsilon )) \leqslant P(\omega ) \leqslant \exp ( - n(H - \varepsilon )).\]
        (3)得证。
	\end{thm}
	\begin{prob} [Discrete Memoryless Source(DMS)]
		\mbox{}\par
		\noindent 对于各类分布$P$的离散无记忆信源,$(S,P)$中$S$表示字母表,$P_i$表示每个字母出现的概率,$S^k$表示长为$k$的字符串。\\
		对于字符串的压缩与解压用过程(code)$(f,\varphi)$表示:\\
		$${S^k}\underset{\varphi }{\overset{f}{\longleftrightarrow}}{{\text{\{ 0,1\} }}^n}$$
		定义${e_{rror}}(f,\varphi ): = {P_k}(\varphi  \circ f(\omega ) \ne \omega ),\omega  \in {\Omega _k}$($\Omega_k$定义同上文).\\
		试想:如果${\text{|S}}{{\text{|}}^k} < {2^n}$,则必存在一一映射使得压缩解压过程不会出错。为了进一步压缩,$n$应尽可能小,即允许出错。\\
		故目标为最小化$\frac{n}{k}$及${e_{rror}}(f,\varphi )$.\\
		\sol
		\mbox{}\par
		\noindent 对于给定$\varepsilon>0$,令$n(k,\varepsilon)$表示满足$e_{rror}(f,\varphi)\le \varepsilon$最小的$n$,下面证明:
		\[\mathop {\lim }\limits_{k \to \infty } \frac{{n(k,\varepsilon )}}{k} =  - \sum\limits_i {{P_i}\log {P_i}}=H \]
		于是,存在$(f,\varphi)$,使得$e_{rror}(f,\varphi) \le \varepsilon$,等价于存在$A  \subseteq \Omega_k$使得$P(A)\geqslant (1-\varepsilon),$且$|A| \le 2^n$(代表$A$内映射不会出错).\\
		设$S(k,\varepsilon)$表示满足上述等价条件的集合A的最小基数,由于$|A| \le  2^n$,有:\\
		$$\biggl\lceil\ logS(k,\varepsilon)\biggr\rceil =n(k,\varepsilon),$$
		即$\mathop {\lim }\limits_{k \to \infty } \frac{{n(k,\varepsilon )}}{k} = \mathop {\lim }\limits_{k \to \infty } \frac{{\log S(k,\varepsilon )}}{k}$.\\
		对于任意$\delta>0$,令$B(k,\varepsilon)$表示满足条件$\omega \in \Omega_k$且$exp(-k(H+\delta))\le P(\omega) \le exp(-k(H-\delta))$的$\omega$的集合。\\
		于是由MacMillian Theorem可知:
		$$B(k,\delta ) \supseteq {\Omega _k}^`$$
		于是,
		$$\left\{ {\begin{array}{*{20}{c}}
			{\mathop {\lim }\limits_{k \to \infty } P(B(k,\delta )) \geqslant \mathop {\lim }\limits_{k \to \infty } P({\Omega _k}^`) = 1} \\
			{|B(k,\delta )| \le exp(k(H+\delta))} \\
		 \end{array}} \right.$$
		 由$S(k,\varepsilon)$最小的性质可知,$S(k,\varepsilon ) \leqslant |B(k,\delta )|$,
		 于是,
		 \begin{align*}  
            \overline {\mathop {\lim }\limits_{k \to \infty } } \frac{1}{k} \cdot \operatorname{logS} (k,\varepsilon ) &\leqslant \overline {\mathop {\lim }\limits_{k \to \infty } } \frac{1}{k} \cdot \log |B(k,\delta )| \\  
            &\leqslant H+\delta  \\  
        \end{align*}
		又,对每一个$A \subseteq \Omega_k$,满足$P(A) \geqslant 1-\varepsilon$,有:\\
		$$\mathop {\lim }\limits_{k \to \infty } P(A \cap B(k,\delta )) \geqslant \frac{{1 - \varepsilon }}{2}$$
		于是,

		\begin{align*}  
            |A| &\geqslant |A \cap B(k,\delta )| \\  
            &\geqslant P(A \cap B(k,\delta )) \cdot |B(k,\delta )|  \\
            &\geqslant \frac{{1 - \varepsilon }}{2} \cdot \exp (k(H - \delta ))
        \end{align*}
        故,
        \begin{align*}  
            \mathop {\underline {\lim } }\limits_{k \to \infty } \frac{1}{k} \cdot \log S(k,\varepsilon ) &\geqslant \mathop {\underline {\lim } }\limits_{k \to \infty } \frac{1}{k} \cdot \log (exp(k(H - \delta )) \cdot \frac{{1 - \varepsilon }}{2})\\
            &\geqslant H-\delta
        \end{align*}
		由于$\delta$的任意性,有
		$$\lim \limits_{k \to \infty } \frac{1}{k} \cdot \log S(k,\varepsilon ) = H$$
	\end{prob}


	\begin{prob} [DMS问题推广]
		\mbox{}\par
		\noindent 建立映射$S \to {R^ + }$,即给每一个字母赋上权值,有
		$$M(\omega ) = {M_1}({\omega _1}) \cdot {M_2}({\omega _2}) \ldots {M_k}({\omega _k}),\omega  \in {\Omega _k}$$
		定义$S(k,\varepsilon ): = \min M(A) = min\sum\limits_{\omega  \in A} {M(\omega )} ,A \subseteq {\Omega _k},{P_k}(A) \geqslant (1 - \varepsilon )$\\
		类似地有
		$$\mathop {\lim }\limits_{k \to \infty } (\frac{{\log (S(k,\varepsilon ))}}{k} - {E_k}) = 0$$
		其中${E_k}: = \frac{1}{k} \cdot \sum\limits_{i = 1}^k {\sum\limits_{x \in S} {{P_i}(x) \cdot log(\frac{{{M_i}(x)}}{{{P_i}(x)}})} }$
		%\sol
		%\mbox{}\par

	\end{prob}

	\begin{prob} [统计问题]
		\mbox{}\par
		\noindent 对于概率分布$P = \{ P(x):x \in X\}$及$Q = \{ Q(x):x \in X\}$,在样本空间$X$中抽$k$次,有$\omega  = ({\omega _1},{\omega _2}, \ldots ,{\omega _k})$,根据已知k次事件构成的序列猜测样本空间概率分布是P还是Q。\\
		要求最小化在实际分布为$Q$时猜错的可能性(保证在实际分布为$P$时猜错的可能性小于$\varepsilon$)。即求$$\beta (k,\varepsilon ) = min{Q_k}(A),A \subseteq {\Omega _k},{P_k}(A) \geqslant (1 - \varepsilon )$$
        类似上述证明,有
        $$\lim \frac{1}{k} \cdot \log \beta (k,\varepsilon ) =  - \sum\limits_{x \in X} {P(x)log\frac{{P(x)}}{{Q(x)}}} $$

	\end{prob}

	\begin{prob} [Homework]
		\mbox{}\par
		\noindent 给定集合$K$到$R$熵函数$h \in {R^{{2^{[K]}}}}$,对于随机变量$X_1, X_2, ..., X_k$,使得$h(A) = H({\{ {x_i}\} _{i \in A}})$。\\
		证明$h$满足次模函数性质:
		$$\left\{ {\begin{array}{*{20}{c}}
			{h(\phi ) = 0} \\
			{h(A) \leqslant h(B),A \subseteq B} \\
			{h(A \cup B) + h(A \cap B) \leqslant h(A) + h(B)}
		  \end{array}} \right.$$
		\sol
	\end{prob}



\end{document}