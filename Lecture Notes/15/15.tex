\documentclass{article}

\usepackage{fancyhdr}
\usepackage{extramarks}
\usepackage{amsmath}
\usepackage{amsthm}
\usepackage{amsfonts}
\usepackage{tikz}
\usepackage[plain]{algorithm}
\usepackage{algpseudocode}

\usepackage{graphicx}
\usepackage{subfigure}

\usepackage{listings}
\usepackage{CJK}



\usetikzlibrary{automata,positioning}

%
% Basic Document Settings
%

\topmargin=-0.45in
\evensidemargin=0in
\oddsidemargin=0in
\textwidth=6.5in
\textheight=9.0in
\headsep=0.25in

\linespread{1.1}

\pagestyle{fancy}
\lhead{\hmwkAuthorName}
\chead{\hmwkClass\ (\hmwkClassInstructor\ \hmwkClassTime): \hmwkTitle}
\rhead{}
\lfoot{\lastxmark}
\cfoot{\thepage}

\renewcommand\headrulewidth{0.4pt}
\renewcommand\footrulewidth{0.4pt}

\setlength\parindent{0pt}

%
% Create Problem Sections
%

\newcommand{\enterProblemHeader}[1]{
    \nobreak\extramarks{}{Problem \arabic{#1} continued on next page\ldots}\nobreak{}
    \nobreak\extramarks{Problem \arabic{#1} (continued)}{Problem \arabic{#1} continued on next page\ldots}\nobreak{}
}

\newcommand{\exitProblemHeader}[1]{
    \nobreak\extramarks{Problem \arabic{#1} (continued)}{Problem \arabic{#1} continued on next page\ldots}\nobreak{}
    \stepcounter{#1}
    \nobreak\extramarks{Problem \arabic{#1}}{}\nobreak{}
}

\setcounter{secnumdepth}{0}
\newcounter{partCounter}
\newcounter{homeworkProblemCounter}
\setcounter{homeworkProblemCounter}{1}
\nobreak\extramarks{Problem \arabic{homeworkProblemCounter}}{}\nobreak{}

\newenvironment{homeworkProblem}{
    \section{Problem \arabic{homeworkProblemCounter}}
    \setcounter{partCounter}{1}
    \enterProblemHeader{homeworkProblemCounter}
}{
    \exitProblemHeader{homeworkProblemCounter}
}

%
% Homework Details
%   - Title
%   - Due date
%   - Class
%   - Section/Time
%   - Instructor
%   - Author
%

\newcommand{\hmwkTitle}{Lecture note}
\newcommand{\hmwkDueDate}{May 15, 2018}
\newcommand{\hmwkClass}{Probability Theory}
\newcommand{\hmwkClassTime}{Class 15}
\newcommand{\hmwkClassInstructor}{}
\newcommand{\hmwkAuthorName}{Jingxiao Chen}


%
% Title Page
%

\title{
    \vspace{2in}
    \textmd{\textbf{\hmwkClass:\ \hmwkTitle}}\\
    \normalsize\vspace{0.1in}\small{\hmwkDueDate\ }\\
    \vspace{0.1in}\large{\textit{\hmwkClassInstructor\ \hmwkClassTime}}
    \vspace{3in}
}

\author{\textbf{\hmwkAuthorName} \\
 }
\date{}

\renewcommand{\part}[1]{\textbf{\large Part \Alph{partCounter}}\stepcounter{partCounter}\\}

%
% Various Helper Commands
%

% Useful for algorithms
\newcommand{\alg}[1]{\textsc{\bfseries \footnotesize #1}}

% For derivatives
\newcommand{\deriv}[1]{\frac{\mathrm{d}}{\mathrm{d}x} (#1)}

% For partial derivatives
\newcommand{\pderiv}[2]{\frac{\partial}{\partial #1} (#2)}

% Integral dx
\newcommand{\dx}{\mathrm{d}x}

% Alias for the Solution section header
\newcommand{\solution}{\textbf{\large Solution}}

% Probability commands: Expectation, Variance, Covariance, Bias
\newcommand{\E}{\mathrm{E}}
\newcommand{\Var}{\mathrm{Var}}
\newcommand{\Cov}{\mathrm{Cov}}
\newcommand{\Bias}{\mathrm{Bias}}



\begin{document}

\begin{CJK*}{UTF8}{gbsn}
\maketitle

\pagebreak
\section{1.母函数(Generating function) }
\textbf{定义}

对于数列$a_n$及其母函数$g(z)$
$$(a_n)^{\infty}_{n=0} \to g(z) = \sum^{\infty}_{n=0}a_n z^n$$

利用g(z)求E(x), Var(x)
$$E(x) = \sum n a_n = g'(1)$$
$$E(x^2) - E(x) = g''(1)$$
$$Var(x) = g''(1)+g'(1)-[g'(1)]^2$$
从E(x), Var(x)求g(z)
$$a_j = \frac{1}{j!}g^{(j)}(0)$$

由唯一性定理得左右两边两两对应,这条性质非常重要,我们可以同过g的性质得到E,Var的性质

\textbf{例1}
	$$\sum^{\infty}_{i=1}a_j z_i = \sum^{\infty}_{i=1}b_i z_i <=> a_i = b_i$$
\textbf{思考题}如何证明上式的正确性

\textbf{定义}
卷积
$$c_i = \sum^l_{j=0} a_j b_{j-1}$$

\textbf{性质}
对于随机变量a,b,c
若a,b独立同分布,且有$g_a(z)g_b(z) = g_c(z)$
则$c=a+b$

\textbf{例2}

$P(T=j) = p_j = q^{j-1}p \quad j\geq 1$几何分布
$$
g(z) = \sum^{\infty}_{j=1}p_j z_j = \frac{p}{q}\sum^{\infty}_{j=1}(q\cdot z)^j
 = (\frac{p z}{1-q z})^n
$$

\textbf{例3: 负二项分布(抛硬币问题)}

\textbf{描述}
假设有一组独立的伯努利数列,每次实验有两种结果“成功”和“失败”。每次实验的成功概率是p,失败的概率是1-p。我们得到一组数列,当预定的“非成功”次数达到r次,那么结果为“成功”的随机次数会服从负二项分布
\[
S_n = \sum_{i=1}^n T_i
\]

其母函数
\[
\begin{split}
g_{S_n}(z) 
	& = (g_T(z))^n \\
	& = (\frac{pz}{1-qz})^n \\
	& = \sum^{\infty}_{j=0} \binom{n+j-1}{n-1} \cdots \\
	& = \sum^{\infty}_{k=n} \binom{k-1}{n-1} p^n q^(k-n) z^k \\
\end{split}
\]
由$(\frac{g(z)}{z}^n) = (\frac{p}{1-qz})^n$有
\[
P(s_n=h+j) = \binom{n+j-1}{j}p^n q^j= \binom{-n}{j}p^n(-q)^j
\]

此外,我们利用$g_X(z) = E(z^X)$可以得到很多东西
例如,若x1,x2独立,
\[
E(z^{x1+x2}) = E(z^{x1})E(z^{x2})
\]

\section{2.拉普拉斯变换(Laplace transform)\quad $L_x(\lambda)$}
\textbf{定义}拉普拉斯变换
对于随机变量X,参数$\lambda \in [0,\infty]$,有函数$L_x(\lambda)$
\[
\text{连续:}
L_x(\lambda) = E(e^{-\lambda x})
	= \int^{\infty}_{-\infty}e^{-\lambda u} f(u)du \\
\]

\[
\text{离散:}
L_x(\lambda) = E(e^{-\lambda x}) 
	= \sum^{\infty}_{j=0} a_j e^{j\lambda} 
	\text{(可能不存在)}
\]

若x1,x2独立:

\[
E(e^{-\lambda(x1+x2)}) 
	= E(e^{-\lambda x1})E(e^{-\lambda x2})
\]

\section{3.矩母函数(moment generating function)\quad $M_x(t)$}
\textbf{定义}矩母函数
对于随机变量X,参数t,有函数$M_x(t)$
\[
\begin{split}
M_x(t) 
	&= E(e^{t x}) \\
	&= E(1+tx+\frac{t^2 x^2}{2!}+\cdots) \\
	&= 1+ E(x)t + \frac{E(x^2)}{2!}t^2 + \cdots\\
\end{split}
\]
本质上讲,矩母函数即是矩的母函数

若x1,x2独立:

\[
M_{x_1}(t)M_{x_2}(t) = M_{x_1+x_2}(t)
\]

\section{4.特征函数(characteristic function) $\varphi_x(\theta)$}
\textbf{定义}特征函数
对于随机变量x
\[
\varphi_x(\theta) = E(e^{i\theta x})
\]
我们可以另$t=i\theta$从而与矩母函数进行类比。

由于涉及到傅里叶变换相关知识,在这里不再进行更多的介绍。

\section{5.常见分布}
\textbf{几种常见分布的矩母函数和特征函数}

\begin{tabular}{ccc}
\quad & $M_x(t)$ & $\varphi(t)$ \\
Bernoulli &  $1-p+pe^t$ & $1-p+pe^{it}$\\
Geomatic & $\frac{pe^t}{1-(1-p)e^t}$ & $\frac{pe^{it}}{1-(1-p)e^{it}}$ \\
$Poisson(\lambda)$ & $e^{\lambda (e^t-1)}\ (t<\lambda)$ & $e^{\lambda (e^{it}-1)}$   \\
Normal $N(\mu, s^2)$ & $e^{t\mu + \frac{1}{2}\sigma^2t^2}$ & $e^{it\mu - \frac{1}{2}\sigma^2t^2}$ \\
\end{tabular}

\textbf{正态分布n阶矩}
若$X\sim N(0,1)$
\[
M_x(\theta) = \int^{\infty}_{\infty} e^{\theta x}\varphi (x) dx
\]
\[
\begin{split}
\varphi (x) 
	&= \frac{e^{-\frac{\lambda ^2}{2}}}{\sqrt{2 \pi}} \\
	&= \frac{1}{\sqrt{2 \pi}} \int^{\infty}_{\infty} exp(\theta x - \frac{x^2}{2}) dx \\
	&= e^{\frac{\theta ^2}{2}} \int^{\infty}_{-\infty} \varphi(x-\theta)dx \\
	&= e^{\frac{\theta ^2}{2}} \\
\end{split}
\]
我们可以得到
\[
	1
	+\frac{\theta ^2}{2} 
	+\frac{1}{2!}(\frac{\theta ^2}{2})^2
	+\cdots
	 = 
	\sum^{\infty}_{n=0}\frac{\theta ^n}{n!}E(X^n)
\]
从中取一项$\frac{1}{n!2^n} = \frac{E(X^{2n})}{2n!}$,明显有
\[
\begin{cases}
	E(X^{2n+1}) = 0\\
	E(X^{2n}) = \frac{(2n)!}{n!2^n}
\end{cases}
\]
\textbf{性质}
对于两个独立的随机变量X1,X2
\begin{align*}
	X_1 \sim N(\mu_1, \sigma_1^2) &\\
	X_2 \sim N(\mu_2, \sigma_2^2)
\end{align*}
\[
 \Rightarrow X_1+X_2\sim N(\mu_1 + \mu_2, \sigma_1^2 + \sigma_2^2)
\]
\textbf{证明}
\[
\begin{split}
M_{x1+x2}(\theta) 
	&= M_{x1}(\theta) M_{x2}(\theta) \\
	&= e^{\mu_1\theta +\frac{\sigma_1^2}{2}\theta^2} e^{\mu_2\theta +\frac{\sigma_2^2}{2}\theta^2}\\
	&= e^{(\mu_1 + \mu_2)\theta +\frac{\sigma_1^2 + \sigma_2^2}{2}\theta^2}\\
	&=^{\text{暂记}} M_x(\theta)
\end{split}
\]
由唯一性定理可知$M_{x1+x2}(\theta) = M_x(\theta)$唯一

\section{6.中心极限定理(Central Limit Theorem)}
\textbf{描述}
对于随机变量$X_i \sim X$, $E(X) = \mu$, $Var(X) = \sigma^2$

记$$Z_n = \frac{X_1+\cdots+X_n-n\mu}{\sqrt{n} \sigma}$$

下式成立$$\lim_{n\to \infty} P(a\leq Z_n \leq b) = F(b) - F(a) = \int^{b}_{\infty} \frac{e^{-\frac{\theta^2}{2}} }{\sqrt{2n}}dx$$

\textbf{证明}
对于特征函数$\varphi_{x-\mu}(\theta) = \varphi(\theta)$
则
\[
\begin{split}
\varphi_{Z_n} 
	&= \prod^n_{j=1} \varphi(\frac{\theta}{\sigma \sqrt{n}}) \\
	&= (\varphi(\frac{\theta}{\sigma \sqrt{n}}))^n \\
\end{split}
\]

\textbf{定理} Levy's Continuity Theorem
\[
\varphi'(\theta) = iE((x-\mu)e^{i\theta(x-\mu)})
\]
\[
\varphi''(\theta) = -E((x-\mu)^2-e^{i\theta(x-\mu)})
\]
由上面两式子在0处泰勒展开
\[
\varphi(\theta) = 1+0-\frac{\sigma^2 \theta^2}{2} + \theta^2 h(\theta)
\]
\[
(\varphi'(0) = 0, \quad \varphi''(0)=-\sigma^2)
\]
注意到当$\theta \to 0$有$h(\theta)\to 0$
\[
\begin{split}
\varphi_{Z_n}(\theta) 
	&= \prod^m_{j=1}\varphi(\frac{\theta}{\sigma \sqrt{n}}) \\
	&= (\varphi(\frac{\theta}{\sigma \sqrt{n}}))^n \\
	&= e^{n\log(1-\frac{\theta^2}{2n} +\frac{\theta^2}{n\sigma^2}h(\frac{\theta}{\sigma \sqrt{n}}))}
\end{split}
\]
又由洛必达法则可以得到
\[
\lim_{n\to \infty}\varphi_{Z_n}(\theta) = e^{-\frac{\theta^2}{2}}
\]
由唯一性定理可得,此时趋向于一个分布
\end{CJK*}
\end{document}
