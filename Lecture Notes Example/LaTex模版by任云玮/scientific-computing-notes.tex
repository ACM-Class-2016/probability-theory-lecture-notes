\documentclass[12pt, a4paper]{article}
\usepackage{ctex}

\usepackage[margin=1in]{geometry}
\usepackage{
  color,
  clrscode,
  amssymb,
  ntheorem,
  amsmath,
  listings,
  fontspec,
  xcolor,
  supertabular,
  multirow,
  mathtools,
  mathrsfs,
  enumerate,
  mathrsfs,
  amssymb
}
\definecolor{bgGray}{RGB}{36, 36, 36}
\usepackage[
  colorlinks,
  linkcolor=bgGray,
  anchorcolor=blue,
  citecolor=green
]{hyperref}
% \newfontfamily\courier{Courier}

\theoremstyle{margin}
\theorembodyfont{\normalfont}
\newtheorem{thm}{定理}
\newtheorem{cor}[thm]{推论}
\newtheorem{pos}[thm]{命题}
\newtheorem{lemma}[thm]{引理}
\newtheorem{defi}[thm]{定义}
\newtheorem{std}[thm]{标准}
\newtheorem{imp}[thm]{实现}
\newtheorem{alg}[thm]{算法}
\newtheorem{exa}[thm]{例}
\newtheorem{prob}[thm]{问题}
\DeclareMathOperator{\sft}{E}
\DeclareMathOperator{\idt}{I}
\DeclareMathOperator{\spn}{span}
\DeclareMathOperator*{\agm}{arg\,min}
\newcommand{\pr}{\prime}
\newcommand{\tr}{^\intercal}
\newcommand{\st}{\text{s.t.}}
\newcommand{\hp}{^\prime}
\newcommand{\ms}{\mathscr}
\newcommand{\mn}{\mathnormal}
\newcommand{\tbf}{\textbf}
\newcommand{\mbf}{\mathbf}
\newcommand{\fl}{\mathnormal{fl}}
\newcommand{\f}{\mathnormal{f}}
\newcommand{\g}{\mathnormal{g}}
\newcommand{\R}{\mathbf{R}}
\newcommand{\Q}{\mathbf{Q}}
\newcommand{\JD}{\textbf{D}}
\newcommand{\rd}{\mathrm{d}}
\newcommand{\str}{^*}
\newcommand{\vep}{\varepsilon}
\newcommand{\lhs}{\text{L.H.S}}
\newcommand{\rhs}{\text{R.H.S}}
\newcommand{\con}{\text{Const}}
\newcommand{\oneton}{1,\,2,\,\dots,\,n}
\newcommand{\aoneton}{a_1a_2\dots a_n}
\newcommand{\xoneton}{x_1,\,x_2,\,\dots,\,x_n}
\newcommand\thmref[1]{定理~\ref{#1}}
\newcommand\lemmaref[1]{引理~\ref{#1}}
\newcommand\defref[1]{定义~\ref{#1}}
\newcommand\posref[1]{命题~\ref{#1}}
\newcommand\secref[1]{节~\ref{#1}}
\newcommand\equref[1]{(\ref{#1})}
\newcommand\figref[1]{图 \ref{#1}}
\newcommand\corref[1]{推论~\ref{#1}}
\newcommand\exaref[1]{例~\ref{#1}}
\newcommand\algref[1]{算法~\ref{#1}}
\newcommand{\remark}{\paragraph{评注}}
\newcommand{\example}{\paragraph{例}}
\newcommand{\proof}{\paragraph{证明}}


\title{科学计算 笔记}
\author{任云玮}
\date{}

\begin{document}
\lstset{numbers=left,
  basicstyle=\scriptsize\courier,
  numberstyle=\tiny\courier\color{red!89!green!36!blue!36},
  language=C++,
  breaklines=true,
  keywordstyle=\color{blue!70},commentstyle=\color{red!50!green!50!blue!50},
  morekeywords={},
  stringstyle=\color{purple},
  frame=shadowbox,
  rulesepcolor=\color{red!20!green!20!blue!20}
}
\maketitle
\tableofcontents
\newpage
\input{01-Intro.tex}

\newpage
\input{02-Interpolation.tex}

\newpage
\input{03-Approximation.tex}

\newpage
\input{04-CalcOfVar.tex}

\newpage
\section{数值积分与数值微分}
\subsection{绪论}
  \begin{thm}[N-L公式]
    设$\f$和$F$定义在$[a, b]$上,$F \in \ms{C}[a, b]$且在
    $(a, b)$上成立$F\hp=\f$,则
    \[
      \int_a^b \f(x)\rd x = F(b) - F(a).
    \]
  \end{thm}
  \remark
    N-L公式是求解积分的基本方法. 但是$F(x)$通常是难以求解的,
    所以需要数值方法.

  \begin{thm}[积分第一中值定理]
    \label{thm: 积分第一中值定理}
    设$\f\in\ms{C}[a, b]$,$\g\in\ms{R}[a, b]$且不变号,
    则存在$\xi\in[a, b]$,成立
    \[
      \int_a^b\f(x)\g(x)\rd x = \f(\xi)\int_a^b\g(x)\rd x.
    \]
  \end{thm}
  \proof
    不妨设$\g(x)\ge 0$. 若$\g(x)\equiv 0$,则结论是显然的.
    考虑$\g$不恒为零的情况. 由于$\f\in\ms{C}$,所以$\f$在$[a,b]$
    上可以取到最小值$m$和最大值$M$,则成立
    \[\begin{split}
      & m\g(x) \le \f(x)\g(x) \le M\g(x) \\
      \Rightarrow \quad &
      m\int_a^b\g(x)\rd x \le \int_a^b\f(x)\g(x)\rd x
      \le M\int_a^b\g(x)\rd x \\
      \Rightarrow \quad &
      m \le \frac{\int_a^b\f(x)\g(x)\rd x}{\int_a^b\g(x)\rd x}
      \le M
    \end{split}\]
    由于连续函数有介质性,所以存在$\xi\in[a, b]$成立
    \[
      \f(\xi) = \frac{\int_a^b\f(x)\g(x)\rd x}{\int_a^b\g(x)\rd x}.
    \]
    即成立
    \[
      \int_a^b\f(x)\g(x)\rd x = \f(\xi)\int_a^b\g(x)\rd x.
      \quad\blacksquare
    \]
  \remark
    注意,定理要求$\f\in\ms{C}[a, b]$,这在很多情况下是难以
    满足的. 但是有些时候可以利用证明中的思路,在$\f$不连续的情况
    下得出同样的结论.

  \begin{pos}
    \label{pos: 积分第一中值定理2}
    设$\f\in\ms{C}[a, b]$,$\g\in\ms{R}[a, b]$且不变号,
    $\psi:[a, b]\to[a, b]$,则存在$\xi\in[a, b]$,使得
    \[
      \int_a^b\f(\psi(x))\g(x)\rd x =
      \f(\xi)\int_a^b\g(x)\rd x.
    \]
  \end{pos}
  \remark
    证明同\thmref{thm: 积分第一中值定理}是几乎一样的。

  \begin{pos}[中矩形公式]
    \label{pos: 中矩形公式}
    设$\f$足够光滑,则
    \[
      \int_a^b\f\rd \approx \f\left(\frac{a+b}{2}\right)(b-a).
    \]
    且存在$\xi\in[a, b]$,其误差$R(\f)=\lhs-\rhs$满足
    \[
      R(\f) = \frac{1}{24}(b-a)^3\f^{\pr\pr}(\xi).
    \]
  \end{pos}
  \proof
    \[\begin{split}
      R(\f) &= \int_a^b\f(x)\rd x - (b-a)\f\left( \frac{a+b}{2} \right) = \int_a^b \left[
        \f(x) - \f\left(\frac{a+b}{2}\right)
      \right]\rd x
    \end{split}\]
    在$x = (a+b)/2$处带Lagrange余项Taylor展开,有
    \[\begin{split}
      R(\f) &= \int_a^b\left[ \f\hp\left(\frac{a+b}{2}\right)\left(x - \frac{a+b}{2}\right)
       + \frac{1}{2}\f^{\pr\pr}(\xi(x))\left(x - \frac{a+b}{2}\right)^2\right]\rd x \\
       &= \frac{1}{2}\int_a^b\f^{\pr\pr}(\xi(x))\left(x - \frac{a+b}{2}\right)^2\rd x
    \end{split}\]
    由于$\xi(x)$的连续性是无法保证的,所以无法直接使用积分中值定理. 但是可以
    根据\posref{pos: 积分第一中值定理2}得,存在$\zeta\in[a, b]$,成立
    \[
      R(\f) = \frac{1}{2}\f^{\pr\pr}(\zeta)\int_a^b\left( x-\frac{a+b}{2} \right)^2\rd x
      = \frac{1}{24}(b-a)^3\f^{\pr\pr}(\zeta).\quad\blacksquare
    \]

  \begin{pos}[梯形公式]
    设$\f$足够光滑,则
    \[
      \int_a^b\f(x)\rd x \approx \frac{\f(a)+\f(b)}{2}(b-a)
    \]
    且存在$\xi\in[a, b]$,其误差$R(\f)=\lhs-\rhs$满足
    \[
      R(\f) = -\frac{1}{12}(b-a)^3\f^{\pr\pr}(\xi).
    \]
  \end{pos}
  \remark
    证明同\posref{pos: 中矩形公式}的证明是相似的. 实际上梯形公式
    可以理解为对原有的函数进行线性插值,并用插值结果的积分来近似原函数
    的积分. 按照这一思路推广,即发现可以用插值多项式的积分来近似原来
    函数的积分.

  \begin{pos}[Simpson公式]
    设$\f$足够光滑,则
    \[
      \int_a^b\f(x)\rd x \approx \frac{b-a}{6}
      \left[ \f(a) + 4\f\left(\frac{a+b}{2}\right)+\f(b) \right].
    \]
    存在$\xi\in[a,b]$,其误差$R(\f)=\lhs-\rhs$满足
    \[
      R(\f) = -\frac{(b-a)^5}{2880}\f^{(4)}(\xi).
    \]
  \end{pos}
  \remark
    实际上这是以$a$,$b$,$(a+b)/2$作插值节点作二次多项式插值后
    的函数积分后的结果. 但是若按照之前的思路证明,会发现$g(x)
    =(x-a)(x-(a+b)/2)(x-b)$在$[a, b]$上是不保号的,所以不能直接
    沿用之前的做法. 为了让它保号,我们可以采用Hermite插值,具体证
    法见下. 另外,有人或许会尝试把它拆分到两个区间上,让$\g$分别保号,
    但这样的做法一般来说是错误的,问题出现在最后一步合并两个区间结果的
    时候,权重有可能会出现负值.
  \proof
    设Hermite插值多项式$H\in P_3$满足$H(a) = \f(a)$,$H(b) = \f(b)$,
    $H((a+b)/2)) = \f((a+b)/2)$,$H\hp((a+b)/2) = \f\hp((a+b)/2)$.
    则存在$\xi\in[a, b]$,使其误差满足
    \[
      \f(x) - H(x) = \frac{\f^{(4)}(\xi)}{4!}(x-a)\left(\frac{a+b}{2}\right)^2(x-b).
    \]
    它在$[a,b]$上是保号的. 对$H(x)$积分即可得$\rhs$. 剩下的内容和之前
    的证明是相同的. $\blacksquare$
  \remark
    根据余项公式可以发现,Simpson公式对于$\f\in P_3$都是精确
    成立的,基于此思想,定义代数精度.

  \begin{defi}[代数精度]
    记\footnote{在以后若不特殊说明,此记号都表示在$[a, b]$上的积分. }
    \[
      I(\f) = \int_a^b\f(x)\rd x.
    \]
    若数值积分公式
    \[
      I(\f) \approx Q(\f)
    \]
    在$\f\in P_n$时精确成立,在$\f\in P_{n+1}$时不精确成立,则称该
    求积公式有$n$阶代数精度.
  \end{defi}
  \remark
    代数精度是求积公式精确程度的一个度量,通常希望求积公式有更高的
    代数精度.

  \begin{defi}[机械求积公式]
    \label{defi: 机械求积公式}
    称求积公式为机械求积公式,若它满足
    \[
      I(\f) = \sum_{k=0}^nA_k\f(x_k).
    \]
    称$x_k\in[a, b]$为求积节点,称$A_k$为求积系数,它的选取与被积函数
    无关,至于求积节点有关.
  \end{defi}
  \remark
    可以发现这一节中出现的求积公式都是机械求积公式,但它们的误差各不相同.
    我们希望可以通过恰当地选取求积节点$x_k$和求积系数$A_k$,使得求积公式
    的代数精度尽可能高. \par
    若我们希望机械求积公式有$m$次代数精度,即意味着对于$1,x,\dots,x^m$
    的积分都是精确成立的,即成立非线性方程组
    \[\begin{split}
      &\sum A_k = b-a,\\
      &\sum A_kx_k = \frac{1}{2}(b^2-a^2),\\
      &\cdots\cdots \\
      &\sum A_kx_k^m = \frac{1}{m+1}(b^{m+1}-a^{m+1}).
    \end{split}\]
    这一方程组求解通常是困难的.

  \begin{defi}[插值求积公式]
    给定节点$a\le x_0 < \cdots < x_n \le n$处的函数值$\f(x_i)$,
    构造$n$次Lagrange多项式,则有
    \[
      I(\f) \approx \sum_{k=0}^n\left( \int_a^bl_k\rd x \right)\f(x_k).
    \]
    $l_k$的定义见\thmref{thm: Lagrange插值法}. 其误差满足
    \[
      R(\f) = \frac{\f^{(n+1)}(\xi)}{(n+1)!}\omega_{n+1}(x)
    \]
  \end{defi}

  \begin{defi}[收敛]
    对于求积公式
    \[
      \sum_{k=0}^n A_k\f(x_k) \approx \int_a^b\f\rd x
    \]
    记$h = \max\{x_i-x_{i-1}\}$,若当$h\to 0$时,$\lhs\to\rhs$,
    则称求积公式是收敛的.
  \end{defi}

  \begin{defi}[稳定]
    设$\tilde{\f}_k$是$\f(x_k)$的近似值. 对于任意的$\vep>0$,
    如果存在$\delta>0$,只需$|\tilde{\f}_k - \f(x_k)|<\delta$,
    就成立
    \[
      \left| Q(\f) - Q(\tilde{\f}) \right| =
      \left| \sum_{k=0}^nA_k(\f(x_k) - \tilde{\f}_k) \right| < \vep,
    \]
    则称求积公式是稳定的.
  \end{defi}
  \remark
    一个求积公式稳定,意味着测量误差并不会随着计算而扩大.

  \begin{thm}[稳定性条件]
    如果机械求积公式的系数$A_k>0$,则该求积公式是稳定的.
  \end{thm}

\subsection{Newton-Cotes公式}
  \begin{thm}[Newton-Cotes公式]
    将积分区间$[a, b]$ $n$等分,步长$h=\frac{b-a}{n}$,构造插值型
    求积公式,即
    \[
      Q(\f) = (b-a)\sum_{k=0}^n C_k^{(n)}\f(x_k)
    \]
    称为Newton-Cotes公式,其中$C_k^{(n)}$称为Cotes系数,为
    \[
      C_k^{(n)} = \frac{h}{b-a}\int_0^n \prod_{j=0,\,j\ne k}^n\frac{t-j}{k-j}\rd t
       = \frac{(-1)^{n-k}}{nk!(n-k)!}\int_0^n\prod_{j=0,\,j\ne k}^n(t-j)\rd t
    \]
  \end{thm}
  \remark
    当$n\ge 8$的时候,Cotes系数出现负值,意味着测量误差会随着计算而增大,所以
    $n\ge 8$的Newton-Cotes公式是不用的.

  \begin{thm}[插值型求积公式的代数精度]
    形如
    \[
      Q(\f) = \sum_{k=0}^nA_k\f(x_k)
    \]
    的求积公式有$n$次或以上代数精度的充要条件为,它是插值型的.
  \end{thm}
  \proof
    充分性是显然的. 对于必要性,用$Q(\f)$计算$l_k$来得到$A_k$,即
    \[
      \int_a^bl_k(x)\rd x = \sum_{j=0}^nA_jl_k(x_j) \quad\Rightarrow\quad
      A_k = \int_a^bl_k(x)\rd x.\quad\blacksquare
    \]

  \begin{thm}[偶阶求积公式的代数精度]
    当$n$为偶数时,$n$阶插值型求积公式至少有$n+1$次代数精度.
  \end{thm}

\subsection{复合求积公式}
  即对原积分区间$n$等分,使得每一段长度$h<1$,之后在每一段上
  分别求积。

  \begin{thm}[复合梯形公式余项]
    \[
      T_n(\f) = \frac{h}{2}[\f(a) + 2\sum_{k=1}^{n-1}\f(x_k) + \f(b)]
    \]
    其余项为
    \[
      R_n(\f) = -\frac{b-a}{12}h^2\f^{\pr\pr}(\eta),\quad
      \eta\in[a, b]
    \]
  \end{thm}

  \begin{thm}[复合Simpson公式]
    \[
      S_n = \frac{h}{6}[\f(a) + 4\sum_{k=0}^{n-1}\f(x_{k+1/2})
      + 2\sum_{k=1}^{n-1}\f(x_k) + \f(b)]
    \]
    其余项为
    \[
      R_n(\f) = -\frac{b-a}{180}\left(\frac{h}{2}\right)^4\f^{(4)}(\eta),
      \quad\eta\in[a, b]
    \]
  \end{thm}

\subsection{Gauss求积公式}
  \begin{pos}[求积公式代数精度上限]
    形如下式的求积公式的代数精度至多为$2n+1$.
    \[
      \int_a^b\f(x)\rd x\approx \sum_{k=0}^nA_k\f(x_k).
    \]
  \end{pos}
  \remark
    根据本节之后的构造,可以发现这个上限是取得到的.

  \begin{defi}[Gauss求积公式]
    设
    \begin{equation}
      \label{equ: 带权求积公式}
      I(\f) = \int_a^b\f(x)\rho(x)\rd x \approx \sum_{k=0}^nA_k\f(x_k) = Q(\f).
    \end{equation}
    若\equref{equ: 带权求积公式}有$2n+1$次代数精度,则称其节点$x_k$为Gauss点,
    称该公式为Gauss型求积公式.
  \end{defi}
  \remark
    若要解权函数对应的Gauss求积公式,则只需要取$\f(x) = x^k$,$k=
    0,1,\dots, 2n+1$,解对应的方程组即可. 但由于它是关于$x_k$和
    $A_k$非线性的方程组,所以需要先确定$x_k$,得到关于$A_k$的线性
    方程组.

  \begin{thm}
    节点为$a \le x_0 < \cdots < x_n = b$的带权机械求积公式$Q(\f)$有
    $n+k$($0\le k\le n+1$)次代数精度的充要条件为:
    \begin{enumerate}
      \item $Q(\f)$为插值型求积公式,
      \item 对任意$p\in P_{k-1}$,成立
      \[
        \int_a^b\rho(x)p(x)\omega_{n+1}(x)\rd x = 0
      \]
    \end{enumerate}
  \end{thm}
  \proof
    todo

  \begin{cor}
    \equref{equ: 带权求积公式}有$2n+1$次代数精度,当且仅当
    节点$\{x_i\}$是$n+1$次正交多项式的根.
  \end{cor}
  \remark
    若积分区间为$[a, b]$,一般会先变换到$[-1, 1]$上. 之后的讨论
    中的区间都选取$[-1, 1]$.

  \begin{pos}
    Gauss求积公式的系数全是正的,从而它是稳定的.
  \end{pos}

  \begin{thm}
      Gauss求积公式是收敛的,即
      \[
        \lim_{n\to\infty}\sum_{k=0}^nA_k\f(x_k)
        = \int_a^b\rho\f\rd x.
      \]
  \end{thm}

\subsection{Romberg求积公式}
  \paragraph{动机}
    在利用复合求积公式的时候,如果需要增加精度,则需要再二分一遍区间.
    即增加了节点$x_{k+\frac{1}{2}} = \frac{1}{2}(x_k + x_{k+1})$.
    如果每次增加节点后都需要按照求积公式重新计算,则显然计算量过大并且浪费
    了先前计算好的结果,所以希望可以充分地利用先前的计算结果来简化计算.

  \begin{alg}[Richardson外推方法]
    \label{alg: Richardson外推方法}
    设$Q$为需要计算的值的精确结果,$Q_1(h)$是通过某一算法计算得
    的$Q$的近似,若误差满足
    \begin{equation}
      \label{equ: Q-Q(h)}
      Q-Q_1(h) = c_1h^{p_1} + c_2h^{p_2} + \cdots = o(h^{p_1-1})
    \end{equation}
    其中$0<p_1<p_2<\cdots$与$h$无关. 则令
    \[
      Q_{k+1}(h) = \frac{Q_k(h/2) - 2^{-p_k}Q_k(h)}{1-2^{-p_k}}.
    \]
    其误差满足
    \[
      Q-Q_{k+1}(h) = c_2^*h^{p_2} + c_3^*h^{p_3} + \cdots = o(h^{p_2-1})
    \]
  \end{alg}
  \remark
    Richardson外推方法意味着如果有二分前的结果$Q(h)$
    和二分后的结果$Q(h/2)$,那么可以通过这两个结果做一次外推
    从而得到更高的精度. 并且二分了多少次,就可以外推多少次.
  \proof
    根据\equref{equ: Q-Q(h)},成立
    \[\begin{split}
      Q - Q(h/2) &= c_1\left(\frac{h}{2}\right)^{p_1} + c_2\left(\frac{h}{2}\right)^{p_2} + \cdots\\
      2^{-p_1}(Q-Q(h)) &= c_1\left(\frac{h}{2}\right)^{p_1} + c_22^{-p_1}h^{p_2} + \cdots\\
    \end{split}\]
    将上述两式相减,即成立
    \[\begin{split}
      &Q-Q(h/2) - 2^{-p_1}(Q-Q(h)) = c_2^*h^{p_2} + c_3^*h^{p_3} + \cdots\\
      \Rightarrow\quad&
      Q_2(h) = \frac{Q(h/2) - 2^{-p_1}Q(h)}{1-2^{-p_1}}\quad\blacksquare
    \end{split}\]

  \begin{pos}
    复合梯形公式外推一次即得Simpson公式,
  \end{pos}

  \begin{pos}[复合梯形公式余项]
    \label{pos: 复合梯形公式余项}
    对于$[a, b]$进行$n$等分,令$h = (b-a)/n$,考虑复合梯形求积公式
    \[
      T(h) = h\left[ \frac{\f(x_0)}{2} + \f(x_1)
      +\cdots + \f(x_{n-1}) + \frac{\f(x_n)}{2} \right].
    \]
    存在与$h$无关的$a_i$,成立
    \[
      T(h) = \int_a^b\f\rd x + \sum_{k=1}^\infty a_{2k}h^{2k}.
    \]
  \end{pos}

  \begin{alg}[Romberg算法]
    记$n$次二等分第$m$次外推后的结果是$R(n, m)$,记
    $h_n = \frac{1}{2^n}(b-a)$,则
    \[\begin{split}
      R(0, 0) &= h_1(\f(a) + \f(b)) \\
      R(n, 0) &= \frac{1}{2}R(n-1, 0)
      + h_n\sum_{k=1}^{2^{n-1}}\f(a+(2k-1)h_n) \\
      R(n, m) &= \frac{1}{4^m-1}(
        4^mR(n,m-1) - R(n-1,m-1)
      )
    \end{split}\]
    其余项是$O(h_n^{2(m-1)})$的.
  \end{alg}
  \remark
    $R(n, 0)$是二等分$n$次后的复合梯形公式的结果,它可以通过
    重复利用$R(n-1, 0)$中的结果来较快得到. 由于
    $R(n, m-1)$和$R(n-1, m-1)$的余项满足\algref{alg: Richardson外推方法}
    中的要求,从而可以进行$n-1$次外推.

\subsection{自适应求积公式}
  \begin{defi}[后验估计]
    如果误差的上界是可计算量,则称为\tbf{后验估计},否则
    则称为\tbf{先验估计}.\footnote{见\exaref{exa: 刘徽割圆术}.}
  \end{defi}

  \begin{exa}[刘徽割圆术]
    \label{exa: 刘徽割圆术}
    设圆的面积为$S$,它是一个不可计算量. 设圆的内接正$n$变形面积
    为$S_n$,它是一个可计算量. 则成立刘徽不等式
    \[
      S_{2n} < S < S_{2n} + (S_{2n} - S_n)
      \quad\Rightarrow\quad
      0 < S - S_{2n} < S_{2n} - S_n.
    \]
    它是对于割圆术误差的一个后验估计.
  \end{exa}

  \begin{thm}[Simpson公式的后验估计]
    记$S_n(a, b)$为对$[a, b]$区间进行$n-1$次二等分,在每一区间上利用
    Simpson公式数值积分的结果,记$I(a, b)$为在$[a, b]$上积分$\f$
    的结果,则成立
    \[
      \left| I(a, b) - S_2(a, b) \right|
      \le \frac{1}{15}| S_2(a, b) - S_1(a, b)|.
    \]
  \end{thm}
  \proof
    Simpson公式的误差满足
    \begin{equation}
      \label{equ: Simpson公式误差}
      I(a, b) - S_1(a, b) = -\frac{b-a}{180}\left(\frac{h}{2}\right)^2\f^{(4)}(\xi)
    \end{equation}
    其中$h = b-a$,$\xi\in[a, b]$. 显然它是一个先验估计. 将$[a, b]$二等分,
    在每段上利用Simpson公式计算,则有
    \[\begin{split}
      I\left(a, \frac{a+b}{2}\right) - S_1\left(a, \frac{a+b}{2}\right)
      & = -\frac{b-a}{360}\left(\frac{h}{4}\right)^2\f^{(4)}(\eta_1) \\
      I\left(\frac{a+b}{2}, b\right) - S_1\left(\frac{a+b}{2}, b\right)
      & = -\frac{b-a}{360}\left(\frac{h}{4}\right)^2\f^{(4)}(\eta_2)
    \end{split}\]
    可以发现两个等式的右侧系数是同号的,所以将它们相加,可得
    \begin{equation}
      \label{equ: Simpson2公式误差}
      I(a, b) - S_2(a, b) = -\frac{b-a}{180}\left(\frac{h}{4}\right)^2\f^{(4)}(\eta)
    \end{equation}
    我们可以假设$h$足够的小,从而可以有$\xi\approx\eta$,从而根据
    \equref{equ: Simpson公式误差}和\equref{equ: Simpson2公式误差},有
    \[\begin{split}
       &16(I(a, b) - S_2(a, b)) \approx I(a, b) - S_1(a, b) \\
      \Rightarrow\quad& |I(a, b) - S_2(a, b)| \le \frac{1}{15}|S_2(a, b) - S_1(a, b)|.
      \quad\blacksquare
    \end{split}\]

  \begin{thm}[停机准则]
    设所要满足的精度为$\vep$,即与精确值间误差小于等于$\vep$,
    则可设置停机准则为
    \[\begin{split}
      |I(a, b) - S_2(a, b)| &\le \vep \\
      |I(c, d) - S_2(c, d)| &\le \frac{d-c}{b-a}\vep.
    \end{split}\]
  \end{thm}

  \begin{alg}[自适应方法]
    首先利用Simpson公式的后验估计和停机准则,选出划分
    $a = x_0 < \cdots x_n = b$. 再用Romberg计算各个子区间
    上的满足一定精度要求的积分值,最后相加. \par
  \end{alg}

\subsection{数值微分}
  \begin{defi}[插值型数值微分]
    \label{defi: 插值型数值微分}
    给定函数$\f(x)$在$a=x_0 < \cdots < x_n = b$处的函数值
    $\f(x_k)$,设$P_n(x)$为该插值节点的$n$次插值多项式,用
    插值多项式在$x_k$处的微分来近似原函数在该点的微分,即
    \[
      P_n\hp(x_k) \approx \f\hp(x_k).
    \]
    注意,我们仅仅考虑插值节点处的导数.
  \end{defi}

  \begin{defi}[含重节点的差商]
    定义
    \[
      \f[x_0,\dots,x_n,x_n] = \lim_{\Delta x\to0}
      \f[x_0,\dots,x_n,x_n+\Delta x].
    \]
  \end{defi}
  \remark
    根据定义,可以用差商来表示差商的导数,即
    \[
      \frac{\rd}{\rd x}\f[x_0,\dots,x_n,x]
      = \f[x_0, \dots, x_n, x, x].
    \]

  \begin{thm}[插值型数值微分的误差]
    设符号同\defref{defi: 插值型数值微分},则在节点$x_k$处的
    误差满足
    \[
      \f\hp(x_k) - P_n(x_k) = \frac{\f^{(n+1)}(\xi)}{(n+1)!}
      \omega\hp_{n+1}(x_k),\quad \xi\in[a, b].
    \]
  \end{thm}
  \proof
    \[\begin{split}
      &\f(x) = P_n(x) + \f[x_0,\dots,x_n,x]\omega_{n+1}(x)\\
      \Rightarrow\quad&
      \f\hp(x) = P\hp_n(x) + \left\{
        \f[x_0,\dots,x_n,x,x]\omega_{n+1}(x)
        + \f[x_0,\dots,x_n,x]\omega\hp_{n+1}(x)
      \right\}.
    \end{split}\]
    令$x=x_k$,有$\omega_{n+1}(x_k) = 0$,从而成立
    \[
      \f\hp(x_k)-P_n\hp(x_k) = \f[x_0,\dots,x_n,x]\omega\hp_{n+1}(x)
      = \frac{\f^{(n+1)}(\xi)}{(n+1)!}\omega_{n+1}\hp(x_k).
      \quad\blacksquare
    \]

  \begin{cor}[中点方法]
    设有$x_{k-1} = x_k - h$,$x_k$,$x_{k+1}=x_k+h$,则
    \[
      \f\hp(x_k) \approx \frac{\f(x_{k+1}) - \f(x_{k-1})}{2h}.
    \]
    其余项满足
    \[
      R(\f, x_k) = \frac{\f^{(3)}(\xi)}{6}h^2.
    \]
  \end{cor}

  \begin{pos}[待定系数法]
    考虑数值微分公式
    \[
      \f\hp(x_k) = \alpha_i\sum_{i=0}^n\f(x_i).
    \]
    利用待定系数法确定$\{\alpha_i\}$,使得$\rhs-\lhs$的余项
    阶数尽可能高.
  \end{pos}

  \begin{pos}[外推]
    中点方法的余项满足Ricahrdson外推的条件,所以可以有
    \[  \begin{split}
      G_0(h) &= \frac{\f(x+h) - \f(x-h)}{2h}, \\
      G_m(h) &= \frac{4^mG_{m-1}(h/2) - G_{m-1}(h)}{4^m-1}.
    \end{split}\]
  \end{pos}

  \begin{exa}[测量误差的影响]
    对于函数$y=\f(x)$,设测量得的函数值为$y_n=\f(x)+\frac{1}{n^2}
    \sin(n^4x)$,则测量误差满足$\|y-y_n\|_{\infty}=\frac{1}{n^2}$,
    但是对于误差,成立
    \[
      \|y\hp-y\hp_n\|_{\infty}
      = \|\frac{1}{n^2}n^4\cos(n^4x)\|_{\infty} = n^2.
    \]
    可以发现节点越多,误差可能反而越大,这说明了求导运算是不稳定的.
  \end{exa}

  \paragraph{todo}
    测量误差与误差分析.


\newpage
\section{非线性方程求根}
\subsection{二分法}
  \begin{thm}[闭区间套定理]
    设$\{[a_n, b_n]\}$满足
    \begin{enumerate}
      \item $[a_n, b_n] \subset [a_{n-1}, b_{n-1}]$,
      \item 当$n\to\infty$时,$|b_n - a_n|\to 0$.
    \end{enumerate}
    则存在$\xi$成立
    \[
      \bigcap_{k=0}^{\infty}[a_n, b_n] = \{\xi\}.
    \]
  \end{thm}

  \begin{thm}[连续函数零点定理]
    \label{thm: 连续函数零点定理}
    设$\f\in\ms{C}[a, b]$且$\f(a)\f(b)<0$,则存在$c\in[a, b]$,
    成立$\f(c) = 0$.
  \end{thm}
  \remark
    这一定理是多种方程求根方法的基础,下给出构造性的证明,这一证明
    本身实际上描述了二分法求根的过程.
  \proof
    构造如下闭区间套,令$[a_0, b_0]=[a, b]$,$c_n=(a_n+b_n)/2$,
    如果$\f(c_n)=0$,则结论成立,否则$c_n$至少与$a_n$和$b_n$中的
    一个异号,取该半个区间为$[a_{n+1}, b_{n+1}]$. 由上述构造可知
    \[
      [a_{n+1}, b_{n+1}]\subset[a_n,b_n],\quad
      |b_n - a_n| = \frac{b-a}{2^n}\to 0.
    \]
    从而存在$\xi\in\bigcap_{k=0}^\infty[a_n, b_n]$.
    若$\f(\xi)\ne 0$,不妨设$\f(\xi) = r>0$,则存在$\delta>0$,
    在$[\xi-\delta,\xi+\delta]$上$\f(x)>0$恒成立. 取足够大的$n$,
    即可使$[a_n,b_n]\subset[\xi-\delta,\xi+\delta]$,其中
    $\f(a_n)\f(b_n)<0$,与$\f$在$[\xi-\delta,\xi+\delta]$保号
    矛盾,从而$\f(\xi)=0$. $\blacksquare$

  \begin{alg}[二分法求解零点]
    对区间$n$等分,对每个满足端点函数值异号的区间$[a_k,a_{k+1}]$,
    按照\thmref{thm: 连续函数零点定理}证明中的方法二分,直到满足
    $b_n-a_n<\vep$,即与零点误差小于$\vep$为止.
  \end{alg}
  \remark
    二分法实现简单,但是在高维情况下,由于不再有“区间端点”的概念,
    所以难以推广.

\subsection{不动点法}
  \begin{defi}[压缩映射]
    设$\f$将$E\subset\R$映射到$E$上. 称$\f$为压缩映射,
    若存在常数$l<1$,对任意$x,y\in E$成立
    \[
      |\f(x) - \f(y)| \le l|x-y|.
    \]
  \end{defi}

  \begin{thm}[压缩映射定理]
    \label{thm: 压缩映射定理}
    设$\varphi$是$[a,b]$上的连续压缩映射,则存在唯一的$x\in[a, b]$,
    成立$\varphi(x) = x$.
  \end{thm}
  \proof
    对于唯一性. 设成立$x_1=\varphi(x_1)$,$x_2=\varphi(x_2)$,则
    \[
      |x_1-x_2| = |\varphi(x_1)-\varphi(x_2)|\le l|x_1-x_2|
      \quad\Rightarrow\quad x_1=x_2.
    \]
    对于存在性. 令$F(x)=x-\varphi(x)\in\ms{C}[a, b]$. 若
    $a=\varphi(a)$或$\varphi(b)$,则得证. 否则由于$\f$映射到$[a,b]$
    自身,所以成立$a<\varphi(a), \varphi(b)<b$. 则成立
    $F(a)F(b)<0$,由\thmref{thm: 连续函数零点定理}可知,存在$\xi\in[a,b]$,
    成立$F(\xi)=0$,即$\xi=\varphi(\xi)$. $\blacksquare$
  \remark
    此定理对于任意的完备度量空间都是成立的,且条件中的连续性要求可以略去,
    证明方法是构造迭代数列$x_{n+1}=\varphi(x_n)$.

  \begin{thm}[余项估计]
    设$\varphi$是$[a, b]$上压缩常数为$l$的连续压缩映射. 则数列
    \[
      x_0 = a,\quad x_{n+1} = \varphi(x_n)
    \]
    收敛至$x_*$. 且有估计式
    \[
      |x_n - x_*| \le \frac{l^n}{1-l}|x_1-x_0|.
    \]
  \end{thm}
  \proof
    下证明$\{x_n\}$为Cauchy序列.
    \[
      |x_{k+1}-x_k| = |\varphi(x_k) - \varphi(x_{k-1})|
      \le l|x_k-x_{k-1}| \le \cdots \le l^k|x_1-x_0|.
    \]
    所以对于任意的$n$和$p>0$,成立
    \[
      |x_{n+p}-x_n|\le \sum_{k=0}^{p-1} |x_{n+k+1}-x_{n+k}|
      \le |x_1-x_0|\sum_{k=0}^{p-1}l^{n+k}.
    \]
    由于$l>0$,根据几何级数的性质,成立
    \[
      |x_{n+p}-x_n| \le \frac{l^n}{1-l}|x_1-x_0|.
    \]
    当$n\to\infty$时,$\rhs\to 0$,从而$\{x_n\}$是Cauchy序列,
    所以收敛. 令$p\to\infty$,即得估计式
    \[
      |x_n-x_*| \le \frac{l^n}{1-l}|x_1-x_0|.
      \quad\blacksquare
    \]

  \begin{defi}[局部收敛]
    设$\varphi(x)$有不动点$x_*$,则对于迭代法
    \begin{equation}
      \label{equ: 迭代法}
      x_{n+1}=\varphi(x_n).
    \end{equation}
    如果存在$x_*$的某个领域$O_\delta(x)$,使得任意$x_0\in O_\delta(x)$,
    \equref{equ: 迭代法}收敛至$x_*$,则称\equref{equ: 迭代法}局部收敛.
  \end{defi}

  \begin{thm}[局部收敛的条件]
    设$x_*$是$\varphi\in\ms{C}^1$的不动点,且$|\varphi\hp(x_*)|<1$,
    则\equref{equ: 迭代法}在$x_*$处局部收敛.
  \end{thm}
  \proof
    由于$\varphi\hp$是连续的,所以存在$O_\delta(x_*)$,对任意的$x\in
    O_\delta(x_*)$,成立
    \[
      |\varphi\hp(x)| < l < 1.
    \]
    从而对于任意的$x,y\in O_\delta(x_*)$,根据中值定理,成立
    \[
      |\varphi(x) - \varphi(y)| = |\varphi\hp(\xi)||x-y| < l|x-y|.
    \]
    即$\varphi(x)$在$O_\delta(x_*)$上是压缩映射,从而对任意$x_0\in
    O_\delta(x_*)$,\equref{equ: 迭代法}收敛至$x_*$. $\blacksquare$

  \begin{pos}[不局部收敛的条件]
    设$x_*$是$\varphi(x)\in\ms{C}^1[a, b]$的不动点,若
    $\varphi(x)$在$[a, b]$上单调且$|\varphi\hp(x_*)|\ge h>1$,则
    $\varphi(x)$不局部收敛.
  \end{pos}
  \proof
    由于$\varphi\hp(x_*)>1$且连续,所以存在$\delta>0$,使得在$O_\delta(x_*)$
    中成立
    \[
      |\varphi(x)-x_*| = |\varphi\hp(\xi)(x-x_*)| \ge h|x-x_*|.
    \]
    所以对任意的$0<\vep<\delta$,对任意的$x_0\in O_\delta(x_*)
    \backslash\{x_*\}$,迭代
    足够多次后成立$|x_n-x_*|>\vep$. 由于$\varphi(x)$单调,所以仅有
    $x=x_*$满足$\varphi(x)=x_*$. 从而若存在$x_0\in[a, b]$,使迭代数列收敛
    至$x_*$,则对任意$\vep>0$,存在$n$,使得$x_n\in O_{\vep}(x_*)\backslash\{x_*\}$.
    对于这样的$n$,继续迭代足够多次后,成立$|x_m-x_*|>\vep$,与收敛矛盾,从而不存在
    $x_0\in[a, b]$,使得迭代数列收敛. $\blacksquare$

  \begin{defi}[$p$阶收敛]
    \label{defi: p阶收敛}
    设$x_*$是$\varphi(x)$的不动点,记迭代误差$e_k=x_k-x_*$,若
    当$k\to\infty$时,成立
    \[
      \frac{e_{k+1}}{e_k^p} \to C \ne 0.
    \]
    则称\equref{equ: 迭代法} $p$阶收敛.
  \end{defi}
  \remark
    此定义描述了迭代式收敛的速度.

  \begin{thm}[$p$阶收敛条件]
    \label{thm: p阶收敛条件}
    设$x_*$是迭代过程$x_{n+1}=\varphi(x_n)$的不动点,若对正整数$p$,
    $\varphi^{(p)}$在$x_*$附加连续,且成立
    \[\begin{split}
      \varphi\hp(x_*) = \cdots = \varphi^{(p-1)}(x_*) = 0,\quad
      \varphi^{(p)}(x_*) \ne 0.
    \end{split}\]
    则\equref{equ: 迭代法}在$x_*$附近$p$阶收敛.
  \end{thm}
  \proof
    在$x_*$处将$\varphi$ Taylor展开即可. 存在
    $\xi$在$x_k$和$x_*$之间,成立
    \begin{equation}
      \label{equ: 导数为零情况余项}
      \frac{x_{k+1}-x_*}{(x_k-x_*)^p} = \frac{\varphi^{(p)}(\xi)}{p!}.
      \quad\blacksquare
    \end{equation}
  \remark
    经实践证明,由于当$k\to\infty$时,$\xi\to x_*$,所以对于前$p-1$阶导数
    为零的迭代式,使用\equref{equ: 导数为零情况余项}来计算\defref{defi: p阶收敛}中的常数是十分方便的.


  \begin{alg}[不动点法]
    对于方程$\f(x) = 0$,将其变形成等价的$x=\varphi(x)$的形式,
    且$\varphi$满足在零点处局部收敛,则可以利用迭代数列
    $x_{n+1} = \varphi(x_n)$来求解方程的根.
  \end{alg}

  \begin{alg}[Aitken $\Delta^2$加速法]
    设$\{x_n\}$为一收敛的迭代数列,令
    \[
      \bar{x}_{k+1} = x_k - \frac{(x_{k+1}-x_k)^2}{x_k-2x_kx_{k+1}+x_{k+2}}
      = x_k - \frac{(\Delta x_k)^2}{\Delta^2x_k}.
    \]
    $\{\bar{x}_k\}$收敛得比$\{x_k\}$更快,即满足
    \[
      \lim_{k\to\infty}\frac{\bar{x}_{k+1}-x_*}{x_k-x_*} = 0.
    \]
  \end{alg}
  \remark
    这一加速法的思路在于
    \[
      x_{n+1}-x_* = \varphi(x_n) - \varphi(x_*) = \varphi\hp(\xi)(x_n-x_*).
    \]
    当区间足够小以至$\varphi\hp$变化不大时,可以近似地将$\varphi\hp(\xi)$看作常量.
    另外,这一做法实际上只是数据的后处理,实际上如果我需要的是$\bar{x}_{n+1}$,那无需计算
    $\bar{x}_1$,$\dots$,$\bar{x}_n$,而只需利用$x_{n}$,$x_{n+1}$,$x_{n+2}$
    来计算$\bar{x}_{n+1}$即可.

  \begin{alg}[Steffensen迭代法]
    \[\begin{cases}
      y_k = \varphi(x_k),\quad z_k = \varphi(y_k), \\
      x_{k+1} = x_k - \dfrac{(y_k-x_k)^2}{z_k-2y_k+x_k}.
    \end{cases}\]
  \end{alg}
  \remark
    这一算法实际上只是将Aitken加速法中的$\bar{x}_n$也加入了
    迭代序列中.

  \begin{thm}[Steffensen迭代法]
    若$x_*$是迭代函数
    \[
      \psi = x - \frac{(\varphi - x)^2}{\varphi\circ\varphi - 2\varphi + x}
    \]
    的不动点,则$x_*$也是$\varphi$的不动点. 反之,若$x_*$是$\varphi$的不动点,
    $\varphi^{\pr\pr}$存在,且$\varphi\hp(x_*)\ne 1$,则$x_*$也是$\psi$的
    不动点,且Steffensen迭代法二阶收敛.
  \end{thm}
  \remark
    这一定理表明,就算$\varphi$所对应的迭代序列不收敛,但是对应
    的Steffensen迭代法的迭代序列还是有可能收敛的.

\subsection{Newton法}
  \begin{thm}[Newton法]
    设$x_*$是$\f(x)$的零点且$\f\hp(x_*)\ne 0$,则
    \[
      \varphi(x) = x - \frac{\f(x)}{\f\hp(x)}
    \]
    当$\f\hp(x_*)\ne 0$时,在$x_*$处局部平方收敛,否则线性收敛.
  \end{thm}
  \remark
    Newton法的动机在于在零点附近,用一个线性函数来近似原函数,即
    \[
      \f(x) \approx \f(x_k) + \f\hp(x_k)(x-x_k) \quad\Rightarrow\quad
      x_{k+1} = x = x_k - \frac{\f(x_k)}{\f\hp(x_k)}.
    \]
    则该线性方程的解是原方程的一个近似解. 证明只需利用
    \thmref{thm: p阶收敛条件}即可.

  \begin{alg}[Newton法]
    Newton法求根的步骤如下:
    \begin{enumerate}
      \item 选定初值$x_0$,计算$\f(x_0)$和$\f\hp(x_0)$;
      \item 按照公式$x_{n+1} = x_n - \f_n/\f\hp_n$,并
            计算$\f(x_{n+1})$和$\f\hp(x_{n+1})$;
      \item 如果$x_n$满足$|\delta|<\vep_1$或$|\f|<\vep_2$,则
        终止迭代,以$x_n$作为所求根,否则进入步骤4. 此处
        $\vep_1$,$\vep_2$是允差,而
        \[
          \delta =
          \begin{cases}
            |x_{n}-x_{n-1}|, & |x_1|<C \\
            \dfrac{|x_n-x_{n-1}|}{x_n}, |x_1|\ge C
          \end{cases}
        \]
        其中$C$是取绝对误差或相对误差的控制常数,一般取$C=1$.
      \item 如果迭代次数达到预先规定的$N$,或$\f\hp_n=0$,则方法
        失败,否则继续迭代.
    \end{enumerate}
  \end{alg}

  \begin{thm}[重根情况的Newton法]
    设$x_*$是$\f(x)$的$m$重根,则根据
    \[
      \varphi(x) = x - m \frac{\f(x)}{\f\hp(x)}
    \]
    所得的迭代序列,在$x_*$附近平方收敛.
  \end{thm}

  \begin{thm}[重根情况的Newton法]
    设$x_*$是$\f(x)$的根,则根据
    \[
      \varphi(x) = x - \frac{\f\f\hp}{(\f\hp)^2 - \f\f^{\pr\pr}}
    \]
    所得的迭代序列在$x_*$附近平方收敛.
  \end{thm}
  \remark
    这一函数实际上是令
    \[
      \mu(x) = \frac{(x-x_*)\g(x)}{m\g(x) + (x-x_*)\g\hp(x)}.
    \]
    在考虑$\varphi(x) = x - \mu/\mu\hp$所得到的. 其缺点在于
    涉及了$\f$的二阶导数.

  \begin{thm}[Newton法的收敛区间]
    设$\f\in\ms{C}^2[a, b]$,且满足$\f(a)\f(b)<0$,在$[a, b]$上
    $\f\hp(x)\ne 0$,且$\f^{\pr\pr}$不变号,以及初值$x_0$使得
    $\f(x_0)\f^{\pr\pr}(x_0)>0$成立. 则Newton法所得序列单调收敛于
    唯一根$x_*$.
  \end{thm}

  \begin{pos}[方程组情况的推广]
    对于$\mbf{F}:\,\R^n\to\R^n$,设$\mbf{x}_0$是它的近似根,
    则可以构造方程组的Newton方法
    \[
      \mbf{x}_{k+1} = \mbf{x}_k - [\mbf{F}\hp(\mbf{x}_k)]^{-1}\mbf{F}(\mbf{x}_k).
    \]
    其中$\mbf{F}\hp$为$\mbf{F}$的Jacobi矩阵.
  \end{pos}
  \remark
    这一算法的最大工作量在于求解Jacobi矩阵的逆,即求解线性方程组.

\newpage
\section{解线性方程组的直接/间接方法}
\subsection{绪论}
  本章讨论的问题是线性方程组
  \begin{equation}
    \label{equ: 线性方程组}
    A\mbf{x} = B,\quad A\in\R^{m\times n},\,\mbf{x}\in\R^n,\,B\in\R^m
  \end{equation}
  的求解. 讨论问题在于
  \begin{enumerate}
    \item \equref{equ: 线性方程组}是否可解.
    \item 若可解,则如何求解.
    \item 若无解,则如何求近似.
  \end{enumerate}
  其中不可解的情况对应$B$为带有误差的的测量数据,求解$X$使得$A\mbf{x}$
  在特定含义下距离$B$尽可能接近. 例如考虑最小二乘法,即求解
  \begin{equation}
    \label{equ: 最小二乘法}
    \mbf{x}^* = \agm \frac{1}{2}\|A\mbf{x}-B\|^2_2
  \end{equation}
  这一问题实际上就是求解极值.

  \begin{thm}[可解性条件]
    设$A=(\alpha_1\,\dots\,\alpha_n)$,其中$\alpha_i$为列向量,
    则以下命题等价
    \begin{enumerate}
      \item $A\mbf{x}=B$有解.
      \item $r(A) = r(\bar{A})$.
      \item $B\in\spn\{\alpha_1,\dots,\alpha_n\}$.
    \end{enumerate}
    其中$r(A)$为$A$的秩,$\bar{A}$为$A$的增广矩阵.
  \end{thm}

  \begin{lemma}[梯度的定价定义]
    \label{lemma: 梯度的等价定义}
    $\f$的梯度$\nabla\f$为对任意方向向量$v$都满足如下条件的函数
    \[
      \langle\nabla\f,\,v\rangle = \JD_v\f = \frac{\rd}{\rd\vep}\f(x+\vep v)
      \bigg|_{\vep=0}.
    \]
  \end{lemma}

  \begin{thm}
    $\f(\mbf{x}) = \frac{1}{2}\|A\mbf{x}-B\|_2^2$的梯度
    可以被表示为
    \[
      \nabla\f(\mbf{x}) = A\tr(A\mbf{x}-B).
    \]
  \end{thm}
  \proof
    下利用\lemmaref{lemma: 梯度的等价定义}证明. 已知
    \[
      \f(\tbf{x}+\vep v) = \frac{1}{2}\langle A\x-B+\vep v,\, A\x-B+\vep v\rangle
      = \frac{1}{2}\|A\x-B\|_2^2 + \vep\langle A\x-B,\,Av \rangle + \frac{1}{2}\vep^2
      \|Av\|_2^2.
    \]
    所以,在该处对任意方向向量$v$成立
    \[\begin{split}
      &\langle\nabla\f(\x),v\rangle = \frac{\rd}{\rd\vep}\f(\x+\vep v)\bigg|_{\vep=0}
      = \langle A\x-B,\,Av\rangle = \langle A\tr(A\x-B),\, v \rangle.
    \end{split}\]
    比较上式的$\lhs$和$\rhs$,可以得
    \[
      \nabla\f = A\tr(A\x-B).\quad\blacksquare
    \]

  \paragraph{todo}
    $m \ll n$情况


\newpage
\section{附录}
\subsection{不等式}

  \begin{lemma}[排序不等式]
    \label{lemma: 排序不等式}
    对于满足下述条件的$\{a_n\}$,$\{b_n\}$,
    \[\begin{split}
      & 0 \le a_1\le a_2\le\cdots\le a_n \\
      & 0 \le b_1\le b_2\le\cdots\le b_n
    \end{split}\]
    则同序相乘求和值最大,逆序最小,即
    \[
      \sum_{i=1}^n a_ib_i \ge \sum_{i=1}^n a_ib_{k_i}
      \ge \sum_{i=1}^n a_ib_{n-i+1}
    \]
  \end{lemma}

  \begin{lemma}[算数-几何均值不等式]
    \[
      (a_1a_2\cdots a_n)^{1/n} \le \frac{a_1+a_2+\cdots+a_n}{n}
    \]
    当且仅当$a_1 = a_2 = \cdots = a_n$时等号成立.
  \end{lemma}
  \proof
    因为有齐次性,所以不妨设$\prod a_i=1$,并令
    \[
      a_1=\frac{\alpha_1}{\alpha_2},\quad
      \dots,\quad
      a_{n-1} = \frac{\alpha_{n-1}}{\alpha_n},\quad
      a_n = \frac{\alpha_n}{\alpha_1}
    \]
    则只需证明下式即可.
    \[
      \frac{\alpha_1}{\alpha_2} + \cdots + \frac{\alpha_n}{\alpha_1}
      \ge n
    \]
    不妨设$\alpha_1 \le \alpha_2 \le \cdots \le \alpha_n$,则根据排序不等式
    \[
      \lhs \ge \alpha_1\frac{1}{\alpha_1} + \cdots + \alpha_n\frac{1}{\alpha_n}
       = n \quad\blacksquare
    \]

\newpage
\subsection{积分相关公式}
  \begin{lemma}[分部积分]
    设$u,v\in\ms{C}^{n+1}[a, b]$,则成立
    \[
      \int_a^buv^{(n+1)}\rd x =
      [ uv^{(n)} - u\hp v^{(n-1)} + \cdots +  (-1)^nu^{(n)}v]
      \bigg\vert_a^b + (-1)^{n+1}\int_a^bu^{(n+1)}v\rd x.
    \]
  \end{lemma}

\newpage
\subsection{特殊函数}
  \begin{defi}[处处连续且不可导]
    称
    \[
      \f(x) = \sum_{n=0}^\infty a^n\cos(b^n\pi x)
    \]
    为Weierstrass函数,其中$0<a<1$,$b$为正奇数,且满足
    \[
      ab > 1 + \frac{3}{2}\pi.
    \]
    它处处连续且处处不可导.
  \end{defi}

\newpage
\input{-3-Euler-Maclaurin.tex}

\end{document}
