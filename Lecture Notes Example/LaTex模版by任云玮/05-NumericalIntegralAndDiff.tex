\section{数值积分与数值微分}
\subsection{绪论}
  \begin{thm}[N-L公式]
    设$\f$和$F$定义在$[a, b]$上,$F \in \ms{C}[a, b]$且在
    $(a, b)$上成立$F\hp=\f$,则
    \[
      \int_a^b \f(x)\rd x = F(b) - F(a).
    \]
  \end{thm}
  \remark
    N-L公式是求解积分的基本方法. 但是$F(x)$通常是难以求解的,
    所以需要数值方法.

  \begin{thm}[积分第一中值定理]
    \label{thm: 积分第一中值定理}
    设$\f\in\ms{C}[a, b]$,$\g\in\ms{R}[a, b]$且不变号,
    则存在$\xi\in[a, b]$,成立
    \[
      \int_a^b\f(x)\g(x)\rd x = \f(\xi)\int_a^b\g(x)\rd x.
    \]
  \end{thm}
  \proof
    不妨设$\g(x)\ge 0$. 若$\g(x)\equiv 0$,则结论是显然的.
    考虑$\g$不恒为零的情况. 由于$\f\in\ms{C}$,所以$\f$在$[a,b]$
    上可以取到最小值$m$和最大值$M$,则成立
    \[\begin{split}
      & m\g(x) \le \f(x)\g(x) \le M\g(x) \\
      \Rightarrow \quad &
      m\int_a^b\g(x)\rd x \le \int_a^b\f(x)\g(x)\rd x
      \le M\int_a^b\g(x)\rd x \\
      \Rightarrow \quad &
      m \le \frac{\int_a^b\f(x)\g(x)\rd x}{\int_a^b\g(x)\rd x}
      \le M
    \end{split}\]
    由于连续函数有介质性,所以存在$\xi\in[a, b]$成立
    \[
      \f(\xi) = \frac{\int_a^b\f(x)\g(x)\rd x}{\int_a^b\g(x)\rd x}.
    \]
    即成立
    \[
      \int_a^b\f(x)\g(x)\rd x = \f(\xi)\int_a^b\g(x)\rd x.
      \quad\blacksquare
    \]
  \remark
    注意,定理要求$\f\in\ms{C}[a, b]$,这在很多情况下是难以
    满足的. 但是有些时候可以利用证明中的思路,在$\f$不连续的情况
    下得出同样的结论.

  \begin{pos}
    \label{pos: 积分第一中值定理2}
    设$\f\in\ms{C}[a, b]$,$\g\in\ms{R}[a, b]$且不变号,
    $\psi:[a, b]\to[a, b]$,则存在$\xi\in[a, b]$,使得
    \[
      \int_a^b\f(\psi(x))\g(x)\rd x =
      \f(\xi)\int_a^b\g(x)\rd x.
    \]
  \end{pos}
  \remark
    证明同\thmref{thm: 积分第一中值定理}是几乎一样的。

  \begin{pos}[中矩形公式]
    \label{pos: 中矩形公式}
    设$\f$足够光滑,则
    \[
      \int_a^b\f\rd \approx \f\left(\frac{a+b}{2}\right)(b-a).
    \]
    且存在$\xi\in[a, b]$,其误差$R(\f)=\lhs-\rhs$满足
    \[
      R(\f) = \frac{1}{24}(b-a)^3\f^{\pr\pr}(\xi).
    \]
  \end{pos}
  \proof
    \[\begin{split}
      R(\f) &= \int_a^b\f(x)\rd x - (b-a)\f\left( \frac{a+b}{2} \right) = \int_a^b \left[
        \f(x) - \f\left(\frac{a+b}{2}\right)
      \right]\rd x
    \end{split}\]
    在$x = (a+b)/2$处带Lagrange余项Taylor展开,有
    \[\begin{split}
      R(\f) &= \int_a^b\left[ \f\hp\left(\frac{a+b}{2}\right)\left(x - \frac{a+b}{2}\right)
       + \frac{1}{2}\f^{\pr\pr}(\xi(x))\left(x - \frac{a+b}{2}\right)^2\right]\rd x \\
       &= \frac{1}{2}\int_a^b\f^{\pr\pr}(\xi(x))\left(x - \frac{a+b}{2}\right)^2\rd x
    \end{split}\]
    由于$\xi(x)$的连续性是无法保证的,所以无法直接使用积分中值定理. 但是可以
    根据\posref{pos: 积分第一中值定理2}得,存在$\zeta\in[a, b]$,成立
    \[
      R(\f) = \frac{1}{2}\f^{\pr\pr}(\zeta)\int_a^b\left( x-\frac{a+b}{2} \right)^2\rd x
      = \frac{1}{24}(b-a)^3\f^{\pr\pr}(\zeta).\quad\blacksquare
    \]

  \begin{pos}[梯形公式]
    设$\f$足够光滑,则
    \[
      \int_a^b\f(x)\rd x \approx \frac{\f(a)+\f(b)}{2}(b-a)
    \]
    且存在$\xi\in[a, b]$,其误差$R(\f)=\lhs-\rhs$满足
    \[
      R(\f) = -\frac{1}{12}(b-a)^3\f^{\pr\pr}(\xi).
    \]
  \end{pos}
  \remark
    证明同\posref{pos: 中矩形公式}的证明是相似的. 实际上梯形公式
    可以理解为对原有的函数进行线性插值,并用插值结果的积分来近似原函数
    的积分. 按照这一思路推广,即发现可以用插值多项式的积分来近似原来
    函数的积分.

  \begin{pos}[Simpson公式]
    设$\f$足够光滑,则
    \[
      \int_a^b\f(x)\rd x \approx \frac{b-a}{6}
      \left[ \f(a) + 4\f\left(\frac{a+b}{2}\right)+\f(b) \right].
    \]
    存在$\xi\in[a,b]$,其误差$R(\f)=\lhs-\rhs$满足
    \[
      R(\f) = -\frac{(b-a)^5}{2880}\f^{(4)}(\xi).
    \]
  \end{pos}
  \remark
    实际上这是以$a$,$b$,$(a+b)/2$作插值节点作二次多项式插值后
    的函数积分后的结果. 但是若按照之前的思路证明,会发现$g(x)
    =(x-a)(x-(a+b)/2)(x-b)$在$[a, b]$上是不保号的,所以不能直接
    沿用之前的做法. 为了让它保号,我们可以采用Hermite插值,具体证
    法见下. 另外,有人或许会尝试把它拆分到两个区间上,让$\g$分别保号,
    但这样的做法一般来说是错误的,问题出现在最后一步合并两个区间结果的
    时候,权重有可能会出现负值.
  \proof
    设Hermite插值多项式$H\in P_3$满足$H(a) = \f(a)$,$H(b) = \f(b)$,
    $H((a+b)/2)) = \f((a+b)/2)$,$H\hp((a+b)/2) = \f\hp((a+b)/2)$.
    则存在$\xi\in[a, b]$,使其误差满足
    \[
      \f(x) - H(x) = \frac{\f^{(4)}(\xi)}{4!}(x-a)\left(\frac{a+b}{2}\right)^2(x-b).
    \]
    它在$[a,b]$上是保号的. 对$H(x)$积分即可得$\rhs$. 剩下的内容和之前
    的证明是相同的. $\blacksquare$
  \remark
    根据余项公式可以发现,Simpson公式对于$\f\in P_3$都是精确
    成立的,基于此思想,定义代数精度.

  \begin{defi}[代数精度]
    记\footnote{在以后若不特殊说明,此记号都表示在$[a, b]$上的积分. }
    \[
      I(\f) = \int_a^b\f(x)\rd x.
    \]
    若数值积分公式
    \[
      I(\f) \approx Q(\f)
    \]
    在$\f\in P_n$时精确成立,在$\f\in P_{n+1}$时不精确成立,则称该
    求积公式有$n$阶代数精度.
  \end{defi}
  \remark
    代数精度是求积公式精确程度的一个度量,通常希望求积公式有更高的
    代数精度.

  \begin{defi}[机械求积公式]
    \label{defi: 机械求积公式}
    称求积公式为机械求积公式,若它满足
    \[
      I(\f) = \sum_{k=0}^nA_k\f(x_k).
    \]
    称$x_k\in[a, b]$为求积节点,称$A_k$为求积系数,它的选取与被积函数
    无关,至于求积节点有关.
  \end{defi}
  \remark
    可以发现这一节中出现的求积公式都是机械求积公式,但它们的误差各不相同.
    我们希望可以通过恰当地选取求积节点$x_k$和求积系数$A_k$,使得求积公式
    的代数精度尽可能高. \par
    若我们希望机械求积公式有$m$次代数精度,即意味着对于$1,x,\dots,x^m$
    的积分都是精确成立的,即成立非线性方程组
    \[\begin{split}
      &\sum A_k = b-a,\\
      &\sum A_kx_k = \frac{1}{2}(b^2-a^2),\\
      &\cdots\cdots \\
      &\sum A_kx_k^m = \frac{1}{m+1}(b^{m+1}-a^{m+1}).
    \end{split}\]
    这一方程组求解通常是困难的.

  \begin{defi}[插值求积公式]
    给定节点$a\le x_0 < \cdots < x_n \le n$处的函数值$\f(x_i)$,
    构造$n$次Lagrange多项式,则有
    \[
      I(\f) \approx \sum_{k=0}^n\left( \int_a^bl_k\rd x \right)\f(x_k).
    \]
    $l_k$的定义见\thmref{thm: Lagrange插值法}. 其误差满足
    \[
      R(\f) = \frac{\f^{(n+1)}(\xi)}{(n+1)!}\omega_{n+1}(x)
    \]
  \end{defi}

  \begin{defi}[收敛]
    对于求积公式
    \[
      \sum_{k=0}^n A_k\f(x_k) \approx \int_a^b\f\rd x
    \]
    记$h = \max\{x_i-x_{i-1}\}$,若当$h\to 0$时,$\lhs\to\rhs$,
    则称求积公式是收敛的.
  \end{defi}

  \begin{defi}[稳定]
    设$\tilde{\f}_k$是$\f(x_k)$的近似值. 对于任意的$\vep>0$,
    如果存在$\delta>0$,只需$|\tilde{\f}_k - \f(x_k)|<\delta$,
    就成立
    \[
      \left| Q(\f) - Q(\tilde{\f}) \right| =
      \left| \sum_{k=0}^nA_k(\f(x_k) - \tilde{\f}_k) \right| < \vep,
    \]
    则称求积公式是稳定的.
  \end{defi}
  \remark
    一个求积公式稳定,意味着测量误差并不会随着计算而扩大.

  \begin{thm}[稳定性条件]
    如果机械求积公式的系数$A_k>0$,则该求积公式是稳定的.
  \end{thm}

\subsection{Newton-Cotes公式}
  \begin{thm}[Newton-Cotes公式]
    将积分区间$[a, b]$ $n$等分,步长$h=\frac{b-a}{n}$,构造插值型
    求积公式,即
    \[
      Q(\f) = (b-a)\sum_{k=0}^n C_k^{(n)}\f(x_k)
    \]
    称为Newton-Cotes公式,其中$C_k^{(n)}$称为Cotes系数,为
    \[
      C_k^{(n)} = \frac{h}{b-a}\int_0^n \prod_{j=0,\,j\ne k}^n\frac{t-j}{k-j}\rd t
       = \frac{(-1)^{n-k}}{nk!(n-k)!}\int_0^n\prod_{j=0,\,j\ne k}^n(t-j)\rd t
    \]
  \end{thm}
  \remark
    当$n\ge 8$的时候,Cotes系数出现负值,意味着测量误差会随着计算而增大,所以
    $n\ge 8$的Newton-Cotes公式是不用的.

  \begin{thm}[插值型求积公式的代数精度]
    形如
    \[
      Q(\f) = \sum_{k=0}^nA_k\f(x_k)
    \]
    的求积公式有$n$次或以上代数精度的充要条件为,它是插值型的.
  \end{thm}
  \proof
    充分性是显然的. 对于必要性,用$Q(\f)$计算$l_k$来得到$A_k$,即
    \[
      \int_a^bl_k(x)\rd x = \sum_{j=0}^nA_jl_k(x_j) \quad\Rightarrow\quad
      A_k = \int_a^bl_k(x)\rd x.\quad\blacksquare
    \]

  \begin{thm}[偶阶求积公式的代数精度]
    当$n$为偶数时,$n$阶插值型求积公式至少有$n+1$次代数精度.
  \end{thm}

\subsection{复合求积公式}
  即对原积分区间$n$等分,使得每一段长度$h<1$,之后在每一段上
  分别求积。

  \begin{thm}[复合梯形公式余项]
    \[
      T_n(\f) = \frac{h}{2}[\f(a) + 2\sum_{k=1}^{n-1}\f(x_k) + \f(b)]
    \]
    其余项为
    \[
      R_n(\f) = -\frac{b-a}{12}h^2\f^{\pr\pr}(\eta),\quad
      \eta\in[a, b]
    \]
  \end{thm}

  \begin{thm}[复合Simpson公式]
    \[
      S_n = \frac{h}{6}[\f(a) + 4\sum_{k=0}^{n-1}\f(x_{k+1/2})
      + 2\sum_{k=1}^{n-1}\f(x_k) + \f(b)]
    \]
    其余项为
    \[
      R_n(\f) = -\frac{b-a}{180}\left(\frac{h}{2}\right)^4\f^{(4)}(\eta),
      \quad\eta\in[a, b]
    \]
  \end{thm}

\subsection{Gauss求积公式}
  \begin{pos}[求积公式代数精度上限]
    形如下式的求积公式的代数精度至多为$2n+1$.
    \[
      \int_a^b\f(x)\rd x\approx \sum_{k=0}^nA_k\f(x_k).
    \]
  \end{pos}
  \remark
    根据本节之后的构造,可以发现这个上限是取得到的.

  \begin{defi}[Gauss求积公式]
    设
    \begin{equation}
      \label{equ: 带权求积公式}
      I(\f) = \int_a^b\f(x)\rho(x)\rd x \approx \sum_{k=0}^nA_k\f(x_k) = Q(\f).
    \end{equation}
    若\equref{equ: 带权求积公式}有$2n+1$次代数精度,则称其节点$x_k$为Gauss点,
    称该公式为Gauss型求积公式.
  \end{defi}
  \remark
    若要解权函数对应的Gauss求积公式,则只需要取$\f(x) = x^k$,$k=
    0,1,\dots, 2n+1$,解对应的方程组即可. 但由于它是关于$x_k$和
    $A_k$非线性的方程组,所以需要先确定$x_k$,得到关于$A_k$的线性
    方程组.

  \begin{thm}
    节点为$a \le x_0 < \cdots < x_n = b$的带权机械求积公式$Q(\f)$有
    $n+k$($0\le k\le n+1$)次代数精度的充要条件为:
    \begin{enumerate}
      \item $Q(\f)$为插值型求积公式,
      \item 对任意$p\in P_{k-1}$,成立
      \[
        \int_a^b\rho(x)p(x)\omega_{n+1}(x)\rd x = 0
      \]
    \end{enumerate}
  \end{thm}
  \proof
    todo

  \begin{cor}
    \equref{equ: 带权求积公式}有$2n+1$次代数精度,当且仅当
    节点$\{x_i\}$是$n+1$次正交多项式的根.
  \end{cor}
  \remark
    若积分区间为$[a, b]$,一般会先变换到$[-1, 1]$上. 之后的讨论
    中的区间都选取$[-1, 1]$.

  \begin{pos}
    Gauss求积公式的系数全是正的,从而它是稳定的.
  \end{pos}

  \begin{thm}
      Gauss求积公式是收敛的,即
      \[
        \lim_{n\to\infty}\sum_{k=0}^nA_k\f(x_k)
        = \int_a^b\rho\f\rd x.
      \]
  \end{thm}

\subsection{Romberg求积公式}
  \paragraph{动机}
    在利用复合求积公式的时候,如果需要增加精度,则需要再二分一遍区间.
    即增加了节点$x_{k+\frac{1}{2}} = \frac{1}{2}(x_k + x_{k+1})$.
    如果每次增加节点后都需要按照求积公式重新计算,则显然计算量过大并且浪费
    了先前计算好的结果,所以希望可以充分地利用先前的计算结果来简化计算.

  \begin{alg}[Richardson外推方法]
    \label{alg: Richardson外推方法}
    设$Q$为需要计算的值的精确结果,$Q_1(h)$是通过某一算法计算得
    的$Q$的近似,若误差满足
    \begin{equation}
      \label{equ: Q-Q(h)}
      Q-Q_1(h) = c_1h^{p_1} + c_2h^{p_2} + \cdots = o(h^{p_1-1})
    \end{equation}
    其中$0<p_1<p_2<\cdots$与$h$无关. 则令
    \[
      Q_{k+1}(h) = \frac{Q_k(h/2) - 2^{-p_k}Q_k(h)}{1-2^{-p_k}}.
    \]
    其误差满足
    \[
      Q-Q_{k+1}(h) = c_2^*h^{p_2} + c_3^*h^{p_3} + \cdots = o(h^{p_2-1})
    \]
  \end{alg}
  \remark
    Richardson外推方法意味着如果有二分前的结果$Q(h)$
    和二分后的结果$Q(h/2)$,那么可以通过这两个结果做一次外推
    从而得到更高的精度. 并且二分了多少次,就可以外推多少次.
  \proof
    根据\equref{equ: Q-Q(h)},成立
    \[\begin{split}
      Q - Q(h/2) &= c_1\left(\frac{h}{2}\right)^{p_1} + c_2\left(\frac{h}{2}\right)^{p_2} + \cdots\\
      2^{-p_1}(Q-Q(h)) &= c_1\left(\frac{h}{2}\right)^{p_1} + c_22^{-p_1}h^{p_2} + \cdots\\
    \end{split}\]
    将上述两式相减,即成立
    \[\begin{split}
      &Q-Q(h/2) - 2^{-p_1}(Q-Q(h)) = c_2^*h^{p_2} + c_3^*h^{p_3} + \cdots\\
      \Rightarrow\quad&
      Q_2(h) = \frac{Q(h/2) - 2^{-p_1}Q(h)}{1-2^{-p_1}}\quad\blacksquare
    \end{split}\]

  \begin{pos}
    复合梯形公式外推一次即得Simpson公式,
  \end{pos}

  \begin{pos}[复合梯形公式余项]
    \label{pos: 复合梯形公式余项}
    对于$[a, b]$进行$n$等分,令$h = (b-a)/n$,考虑复合梯形求积公式
    \[
      T(h) = h\left[ \frac{\f(x_0)}{2} + \f(x_1)
      +\cdots + \f(x_{n-1}) + \frac{\f(x_n)}{2} \right].
    \]
    存在与$h$无关的$a_i$,成立
    \[
      T(h) = \int_a^b\f\rd x + \sum_{k=1}^\infty a_{2k}h^{2k}.
    \]
  \end{pos}

  \begin{alg}[Romberg算法]
    记$n$次二等分第$m$次外推后的结果是$R(n, m)$,记
    $h_n = \frac{1}{2^n}(b-a)$,则
    \[\begin{split}
      R(0, 0) &= h_1(\f(a) + \f(b)) \\
      R(n, 0) &= \frac{1}{2}R(n-1, 0)
      + h_n\sum_{k=1}^{2^{n-1}}\f(a+(2k-1)h_n) \\
      R(n, m) &= \frac{1}{4^m-1}(
        4^mR(n,m-1) - R(n-1,m-1)
      )
    \end{split}\]
    其余项是$O(h_n^{2(m-1)})$的.
  \end{alg}
  \remark
    $R(n, 0)$是二等分$n$次后的复合梯形公式的结果,它可以通过
    重复利用$R(n-1, 0)$中的结果来较快得到. 由于
    $R(n, m-1)$和$R(n-1, m-1)$的余项满足\algref{alg: Richardson外推方法}
    中的要求,从而可以进行$n-1$次外推.

\subsection{自适应求积公式}
  \begin{defi}[后验估计]
    如果误差的上界是可计算量,则称为\tbf{后验估计},否则
    则称为\tbf{先验估计}.\footnote{见\exaref{exa: 刘徽割圆术}.}
  \end{defi}

  \begin{exa}[刘徽割圆术]
    \label{exa: 刘徽割圆术}
    设圆的面积为$S$,它是一个不可计算量. 设圆的内接正$n$变形面积
    为$S_n$,它是一个可计算量. 则成立刘徽不等式
    \[
      S_{2n} < S < S_{2n} + (S_{2n} - S_n)
      \quad\Rightarrow\quad
      0 < S - S_{2n} < S_{2n} - S_n.
    \]
    它是对于割圆术误差的一个后验估计.
  \end{exa}

  \begin{thm}[Simpson公式的后验估计]
    记$S_n(a, b)$为对$[a, b]$区间进行$n-1$次二等分,在每一区间上利用
    Simpson公式数值积分的结果,记$I(a, b)$为在$[a, b]$上积分$\f$
    的结果,则成立
    \[
      \left| I(a, b) - S_2(a, b) \right|
      \le \frac{1}{15}| S_2(a, b) - S_1(a, b)|.
    \]
  \end{thm}
  \proof
    Simpson公式的误差满足
    \begin{equation}
      \label{equ: Simpson公式误差}
      I(a, b) - S_1(a, b) = -\frac{b-a}{180}\left(\frac{h}{2}\right)^2\f^{(4)}(\xi)
    \end{equation}
    其中$h = b-a$,$\xi\in[a, b]$. 显然它是一个先验估计. 将$[a, b]$二等分,
    在每段上利用Simpson公式计算,则有
    \[\begin{split}
      I\left(a, \frac{a+b}{2}\right) - S_1\left(a, \frac{a+b}{2}\right)
      & = -\frac{b-a}{360}\left(\frac{h}{4}\right)^2\f^{(4)}(\eta_1) \\
      I\left(\frac{a+b}{2}, b\right) - S_1\left(\frac{a+b}{2}, b\right)
      & = -\frac{b-a}{360}\left(\frac{h}{4}\right)^2\f^{(4)}(\eta_2)
    \end{split}\]
    可以发现两个等式的右侧系数是同号的,所以将它们相加,可得
    \begin{equation}
      \label{equ: Simpson2公式误差}
      I(a, b) - S_2(a, b) = -\frac{b-a}{180}\left(\frac{h}{4}\right)^2\f^{(4)}(\eta)
    \end{equation}
    我们可以假设$h$足够的小,从而可以有$\xi\approx\eta$,从而根据
    \equref{equ: Simpson公式误差}和\equref{equ: Simpson2公式误差},有
    \[\begin{split}
       &16(I(a, b) - S_2(a, b)) \approx I(a, b) - S_1(a, b) \\
      \Rightarrow\quad& |I(a, b) - S_2(a, b)| \le \frac{1}{15}|S_2(a, b) - S_1(a, b)|.
      \quad\blacksquare
    \end{split}\]

  \begin{thm}[停机准则]
    设所要满足的精度为$\vep$,即与精确值间误差小于等于$\vep$,
    则可设置停机准则为
    \[\begin{split}
      |I(a, b) - S_2(a, b)| &\le \vep \\
      |I(c, d) - S_2(c, d)| &\le \frac{d-c}{b-a}\vep.
    \end{split}\]
  \end{thm}

  \begin{alg}[自适应方法]
    首先利用Simpson公式的后验估计和停机准则,选出划分
    $a = x_0 < \cdots x_n = b$. 再用Romberg计算各个子区间
    上的满足一定精度要求的积分值,最后相加. \par
  \end{alg}

\subsection{数值微分}
  \begin{defi}[插值型数值微分]
    \label{defi: 插值型数值微分}
    给定函数$\f(x)$在$a=x_0 < \cdots < x_n = b$处的函数值
    $\f(x_k)$,设$P_n(x)$为该插值节点的$n$次插值多项式,用
    插值多项式在$x_k$处的微分来近似原函数在该点的微分,即
    \[
      P_n\hp(x_k) \approx \f\hp(x_k).
    \]
    注意,我们仅仅考虑插值节点处的导数.
  \end{defi}

  \begin{defi}[含重节点的差商]
    定义
    \[
      \f[x_0,\dots,x_n,x_n] = \lim_{\Delta x\to0}
      \f[x_0,\dots,x_n,x_n+\Delta x].
    \]
  \end{defi}
  \remark
    根据定义,可以用差商来表示差商的导数,即
    \[
      \frac{\rd}{\rd x}\f[x_0,\dots,x_n,x]
      = \f[x_0, \dots, x_n, x, x].
    \]

  \begin{thm}[插值型数值微分的误差]
    设符号同\defref{defi: 插值型数值微分},则在节点$x_k$处的
    误差满足
    \[
      \f\hp(x_k) - P_n(x_k) = \frac{\f^{(n+1)}(\xi)}{(n+1)!}
      \omega\hp_{n+1}(x_k),\quad \xi\in[a, b].
    \]
  \end{thm}
  \proof
    \[\begin{split}
      &\f(x) = P_n(x) + \f[x_0,\dots,x_n,x]\omega_{n+1}(x)\\
      \Rightarrow\quad&
      \f\hp(x) = P\hp_n(x) + \left\{
        \f[x_0,\dots,x_n,x,x]\omega_{n+1}(x)
        + \f[x_0,\dots,x_n,x]\omega\hp_{n+1}(x)
      \right\}.
    \end{split}\]
    令$x=x_k$,有$\omega_{n+1}(x_k) = 0$,从而成立
    \[
      \f\hp(x_k)-P_n\hp(x_k) = \f[x_0,\dots,x_n,x]\omega\hp_{n+1}(x)
      = \frac{\f^{(n+1)}(\xi)}{(n+1)!}\omega_{n+1}\hp(x_k).
      \quad\blacksquare
    \]

  \begin{cor}[中点方法]
    设有$x_{k-1} = x_k - h$,$x_k$,$x_{k+1}=x_k+h$,则
    \[
      \f\hp(x_k) \approx \frac{\f(x_{k+1}) - \f(x_{k-1})}{2h}.
    \]
    其余项满足
    \[
      R(\f, x_k) = \frac{\f^{(3)}(\xi)}{6}h^2.
    \]
  \end{cor}

  \begin{pos}[待定系数法]
    考虑数值微分公式
    \[
      \f\hp(x_k) = \alpha_i\sum_{i=0}^n\f(x_i).
    \]
    利用待定系数法确定$\{\alpha_i\}$,使得$\rhs-\lhs$的余项
    阶数尽可能高.
  \end{pos}

  \begin{pos}[外推]
    中点方法的余项满足Ricahrdson外推的条件,所以可以有
    \[  \begin{split}
      G_0(h) &= \frac{\f(x+h) - \f(x-h)}{2h}, \\
      G_m(h) &= \frac{4^mG_{m-1}(h/2) - G_{m-1}(h)}{4^m-1}.
    \end{split}\]
  \end{pos}

  \begin{exa}[测量误差的影响]
    对于函数$y=\f(x)$,设测量得的函数值为$y_n=\f(x)+\frac{1}{n^2}
    \sin(n^4x)$,则测量误差满足$\|y-y_n\|_{\infty}=\frac{1}{n^2}$,
    但是对于误差,成立
    \[
      \|y\hp-y\hp_n\|_{\infty}
      = \|\frac{1}{n^2}n^4\cos(n^4x)\|_{\infty} = n^2.
    \]
    可以发现节点越多,误差可能反而越大,这说明了求导运算是不稳定的.
  \end{exa}

  \paragraph{todo}
    测量误差与误差分析.
